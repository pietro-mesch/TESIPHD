\documentclass[10pt,a4paper,twoside]{book}
\usepackage[utf8]{inputenc}
\usepackage{amsmath}
\usepackage{amsfonts}
\usepackage{amssymb}
\usepackage[left=2cm,right=2cm,top=2cm,bottom=2cm]{geometry}
\author{Pietro Meschini}
\title{Forse non muore nessuno neanche stavolta}
\begin{document}

\chapter{La rete viaria urbana}
Formalizzazione, tutta la roba dal libro di Gaetano.

\chapter{Semafori Intelligenti}
\section{Semaforizzazione adattiva}
Perché fare i semafori che si adattano in tempo reale?

Reattività alle condizioni inaspettate nel breve periodo

Reattività agli incidenti

\subsection{Il dilemma dell'ottimizzazione}
In fondo è un modo di migliorare l'offerta per il traffico veicolare privato, causandone l'aumento.

Inoltre incidenti, traffico smodato e imprevedibile sono tutte prerogative del trasporto privato. Si farebbe prima a cercare di curare il bisogno di trasporto privato che ad arginarne gli effetti dannosi.

\section{Tipi di ottimizzazione}
Tipi di ottimizzazione, roba dal libro.


\chapter{Obiettivi Intelligenti}
Definizione degli obiettivi dell'ottimizzazione classica e di quella sostenibile.

Formalizzazione delle funzioni di costo:
l'indice di coda (integrale tempo in coda per veicoli)/tempo passato nel corridoio fa cacare
$$
\frac{•}{•}\sum_{a \in C} \sum_{t \in T}{queu_{at}}}{\sum_{a \in C} \sum_{t \in T}{n_{at}}}
$$

non è reattivo soprattutto se gli archi sono lunghi perché il tempo in coda è comunque poco rispetto al totale

ci vorrebbe tempo per km percorso, ma come faccio a calcolare lo spazio percorso?

\chapter{Ottimizzazione con GLTM Genetico}
Perché è una buona idea?
E' una simulazione in grado di tenere conto di fenomeni complessi ma più veloce della micro.

L'infrastruttura contiene dati in tempo reale etc.

\section{General Link Transmission Model}
Funzionamento generale, pregi e limitazioni rispetto all'applicazione come ottimizzatore.


\section{Il modello in tempo reale}

\section{Tuning del modello}


\section{Tuning del Genetico}
Effetti dimensioni popolazione, aggiustamenti mutazione

\chapter{Risultati}
\section{La qualità in termini assoluti}


\section{Il confronto con microsimulazione}


\section{Performance}

\end{document}