\chapter{Semafori Intelligenti}
\section{Semaforizzazione adattiva}
Perché fare i semafori che si adattano in tempo reale?

Reattività alle condizioni inaspettate nel breve periodo

Reattività agli incidenti

\subsection{Il dilemma dell'ottimizzazione}
In fondo è un modo di migliorare l'offerta per il traffico veicolare privato, causandone l'aumento.

Inoltre incidenti, traffico smodato e imprevedibile sono tutte prerogative del trasporto privato. Si farebbe prima a cercare di curare il bisogno di trasporto privato che ad arginarne gli effetti dannosi.

L'unica ragione per fare ricerca nel campo dell'ottimizzazione semaforica è quella di vedere se attraverso i metodi euristici si riescono ad individuare soluzioni efficaci nel medio e lungo termine 


\section{Tipi di ottimizzazione}
Tipi di ottimizzazione, roba dal libro.

\subsection{BALANCE}

\begin{itemize}
\item Il modello è meglio o peggio di TRE ?
\item Perché usare TRE che è molto più lento?
\item come funziona BAL e cosa può fare?
\end{itemize}

Balance ha un modellino \emph{mesoscopico} tipo gltm (n realtà è proprio come GLTM al secondo) e fa un taglio delle reti intorno alle junction che vuole controllare. Sugli ingressi alle sottoreti (una per ogni junction) usa profili di flusso costanti, ma se può usa i flussi uscenti di una junction per determinare i flussi entranti in una a valle (propagazione).

Il modello viene usato per ricavare le funzioni di costo FERMATE, LUNGHEZZA CODE (in realtà numero di veh in coda) e PERDITEMPO.
Lui vede le code come F-E perché i suoi archi sono in realtà le corsie di svolta, ed usa una lunghezza MASSIMA per le code.

Punti di forza di Balance:
\begin{itemize}
\item Fa tante intersezioni
\item E' veloce
\item Aggiusta anche le durate degli stage
\end{itemize}

Punti deboli:
\begin{itemize}
\item Non è detto che le intersezioni si parlino tanto bene
\item Non guarda avanti
\item non vede l'arco ma solo l'approccio: forse una volta che la coda ha raggiunto il sensore per lui tutte le situazioni sono uguali, e sotto carico non gli cambia più niente
\item Probabilmente tende a massimizzare la capacità dove è più richiesta, favorendo lo scorrimento ma provocando un comportamento "ingordo" che crea problemi a valle.
\end{itemize}

\subsubsection{Requisiti}
Balance lavora stage based, cambiando gli istanti di inizio degli interstage.
Per forzare il comportamento "solo offset" bisogna bloccare la durata degli stage, e lasciare libertà totale a tutti gli interstage.

Serve un detector con channel unico per ogni corsia interessante di approccio all'intersezione.