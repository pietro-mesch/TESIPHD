\chapter{Smart Signals}
Over the years, many attempts have been made to render the signalisation system of urban networks capable of reacting autonomously to the traffic conditions, to address the mutable nature of transportation demand.

In this context, the term \emph{optimisation} is used in its broader sense of \emph{choice of the best option}, whether this is picked out of a set of previously planned solutions, tailored on-the-fly onto the current traffic conditions, or simply the result of a sequence of best possible actions evaluated individually: the most relevant traits of each class of very different approaches will be illustrated in the following sections.

The one thing that all responsive traffic control systems have is the need to perceive the traffic state on the network by means of detectors. The type and amount of information required for different optimisation approaches may vary, but in the end it always boils down to one, or a combination, of the following quantities:
\begin{description}
\item[flow]: the number of vehicles crossing a road section in a given amount of time \units{veh/s}
\item[occupancy]: the share of time during which a road section is occupied by any vehicle \units{\%}
\item[velocity]: the average speed of the vehicles through a road section \units{m/s}
\end{description}
If an adaptive system is to operate the signals effectively, the above quantities must be known for all relevant arcs of the network (or sub-network) that the system is in charge of. Unless otherwise specified, it is assumed throughout this chapter that all required inputs be available and reliable for each of the described methods.

The means for obtaining and processing traffic data are beyond the scope of this work, and the integration with real-time traffic management software allows this separation of tasks; it also guarantees that reasonably reliable traffic data can be obtained for any arc of the network regardless of the physical presence of a detector on a particular road section.


\section{Adaptive Signalisation}
Signalisation of road intersections is necessary because the safety of road users and their fair sharing of the infrastructure cannot be entrusted to their own good sense. The practices used in signal planning aim to produce signal plans that are as efficient as possible, i.e. that minimise the waste of time and infrastructure capacity, under the traffic conditions that can be reasonably expected at each particular junction.

This however, as is evident from the everyday experiences of any driver, implies that often some green time that was allocated for some \emph{potential} flow on a given approach is wasted on an empty street, while on the busiest road vehicles queue fruitlessly at the red light. It may also happen that a road that is not usually busy will temporarily become a main traffic artery because of some special event or accident, rendering the level of priority assigned to it for signal planning completely inadequate.

Adaptive traffic signals represent an attempt to avoid the inevitable inefficiencies of fixed, pre-timed signalisation by responding in real-time to the actual traffic conditions. Their main goals are therefore to:
\begin{itemize}
\item adapt to short term fluctuations in the vehicle arrival pattern, in order to allocate green efficiently on a cycle-to-cycle basis;
\item adapt to unexpected flows deviating from statistical forecast, in order to prioritise approaches according to the real saturation levels;
\item adapt to special events and accidents, and possibly act preemptively to avoid deterioration of the network performance if the occurrence is expected or can be recognised in advance.
\end{itemize}
It should be clear that these objectives, listed here in order of time frame length and complexity, may or may not be met by each class of adaptive signals, but conceptually represent the direction of desired improvements over fixed time signalisation. 

The next few sections present different adaptive signalisation solutions, distinguishing between \emph{actuated} signals that simply respond through a set of rules, and \emph{plan generation} systems, each best suited to address one issue or the other. The specific problems addressed by the approach presented in this work are not dissimilar, as will be detailed in Chapter \ref{c:optimiser}.

\section{Traffic Actuated Signals}


\subsection{Traffic Actuated Control}
The class of traffic control methods referred to as \emph{actuated} generally don’t rely much (if
at all) on an underlying network model, and seldom deal with the very concept of signal
program as anything beyond a predetermined sequence of phases.

In essence, an actuated controller put in charge of a junction inherits the task that
once was a traffic officer’s: by applying a set of rules it attempts to give as much right-of-way
as possible to congested approaches — for as long as it’s needed — while keeping an eye on other
flows in check to avoid having any queue standing for too long. Just like their human
counterparts, actuated signals are extremely effective at maximising the throughput of their
own junction, thanks to the direct gauge of traffic on every approach and very fast reaction
times, but may prove disastrous at the wider network level since a poorly designed control
strategy may introduce self-induced oscillations in the traffic flows, rendering the whole
system unstable — particularly when flows approach critical values.

Signal actuation depends on real time data acquired at the junction by short range sensors
that monitor individual approaches: to this end, cameras have recently started replacing street
level inductive loops, as a single device is often capable of monitoring several approaches.
The first traffic actuated intersection was tested in the USA in 1930. The controller relied
on microphones to detect vehicles waiting on the lesser approaches, and drivers had to honk to signal their presence. Since then, the available technologies have improved, but the simple
fact remains that with cheap electronics and very simple logic (analog friendly, if necessary)
an actuated controller has long been capable of looking after a junction better than any pretimed
plan ever will, no matter how well the timings are optimised to fit \emph{expected} flows.

Different levels of automation are generally classified into two categories:
\begin{description}
\item[semi-actuated]: the controller monitors the \emph{low flow} secondary approaches to allocate them green time only as required, and otherwise serves the main approaches;
\item[actuated]: the controller monitors all approaches and continuously updates the duration of each phase to distribute green time optimally.
\end{description}

\subsubsection*{Principles of Operation}
Every few seconds, an actuated controller must answer the question: \textit{“should the
transition to the next phase start right now?”} or some variant thereof (e.g. to include the
possibility to skip a phase). Actuated junction control is now commonplace around the world:
any given implementation may rely on different types of sensor and data (e.g. cameras or
pressure plates, simple counts or occupancy), but could likely be reduced to the basic
principles (based on common induction loop readings) presented in this section.

Consider a signal phase serving a single approach a to a junction.\\
The incoming lane is equipped with an induction loop a short way back from the stop line
(just enough to remain upstream of the back of the queue for most of the time), through which
the signal controller measures the time interval $t_a^h$ between subsequent vehicle detections,
commonly referred to as \emph{headway}.

After the phase has started, the signal controller determines its duration on-the-fly, based
on sensor readings in relation to a few fundamental parameters such as minimum and
maximum green durations $\gpmin$ and $\gpmax$ and maximum headway $\hat{t}_a^h$. The latter can be considered to represent a minimum flow rate required to extend the phase duration, and does not necessarily concern a single lane group.
In the scope of this example, since only one phase and one lane group are considered, the lane subscript $a$ will be dropped.

The actuation parameters can be fixed, or determined in real time by taking into account the traffic flow on the relevant manoeuvres, the standing queues before the phase start $\quen{a}$, the number of vehicles queuing for lane groups served by other phases $\quen{b \neq a}$, or all of the above.

Between the minimum and maximum green duration values, the junction signal controller
continuously checks whether the time elapsed since the last vehicle passage has exceeded the
maximum headway value for the current phase. If so, the transition to the next phase begins:
\eq{e:phasetrans}{
\text{Initiate phase transition } p \rightarrow p+1 \text{ if }
\begin{cases}
t \geqslant \gpmin \\
t \geqslant \gpmin \vee t^h \geqslant \hat{t}_p^h \\
\end{cases}
}
where the minimum green value can be obtained similarly to \req{e:tclear} in order to at least ensure discharge of the standing queue, possibly using an estimate of the incoming flow, and the
maximum may depend on the minimum green of other manoeuvres and residual cycle time.
After the initial minimum green, the headway threshold can be a dynamic function of flows,
queues and time since phase start $t$, e.g:
\eq{e:dynamicthreshold}{
\hat{t}_p^h (\tau) = \hat{t}^{h0} \left(
1-\tau^{\beta \left( \quen{q \neq p} \right)} \right)
\qquad \text{with} \quad
\tau = \frac{t-\gpmin}{\gpmax - \gpmin}}
whereby the maximum headway decays from its initial value $\hat{t}_a^{h0}$, over the time span between the minimum and maximum green, at a rate determined by any positive nondecreasing
function $\beta$ of the queues accumulated on other approaches $\quen{q \neq p}$.

Note that the independent variable $\tau \in [0,1]$, which means that in case of very high or very
low queues on other approaches the headway threshold drops to zero either right after the minimum
green, or not until the end of the maximum green, respectively.

Furthermore, even as queues on other approaches grow between $\gpmin$ and $\gpmax$, the
maximum threshold remains monotonic nonincreasing as long as $\beta$ is a sensible function of
the (nondecreasing) queues, as seen in Figure \ref{f:densityvolumeactuation}. The basic principles expressed so far in \req{e:phasetrans} and \req{e:dynamicthreshold} can be shaped into any of the most
common categories of actuated control illustrated hereafter.

\textit{Volume actuation:} in the simplest case, the green signal duration is bound between fixed
minimum and maximum design values. It can be extended beyond the minimum value only as
long as vehicles keep reaching the junction at sufficiently short intervals, as seen in Figure \ref{f:volumeactuation}.
Each vehicle arrival starts or resets a timer, and the next phase is initiated as soon as the gap
between subsequent vehicles surpasses the headway threshold $\hat{t}_a^{h}$, which is also constant.
\fig{htbp}{PIX/volume_actuation.png}{f:volumeactuation}{ Volume actuation: on the horizontal axis, time since the start of the current phase; on the vertical axis, time elapsed after each vehicle detection (triangular markers). Each vehicle reaching the sensor before the headway threshold resets the timer. Shaded and black markers respectively represent vehicles reaching the sensor after the maximum headway time has been exceeded, and vehicles that must stop at the red light. }{width=\textwidth}

\textit{Volume-density actuation:} follows the same principles of volume actuation but the
minimum green time is determined by the amount of vehicles initially queuing at the stop
line. The maximum headway allowed to extend the current phase becomes more and more restrictive as the maximum green duration is approached, as portrayed by any of the lightly-shaded curves in Figure \ref{f:densityvolumeactuation}, each corresponding to a different fixed value of $\beta$.

\textit{Density actuation:} the headway threshold decay rate is governed by the number of vehicles detected on the other approaches through the exponent $\beta$, so that at high saturation levels a drop in arrival rate, which denotes the end of a queue or the rear of a dense vehicle platoon, may trigger the transition to the next phase.

\fig{htbp}{PIX/volume_density_actuation.png}{f:densityvolumeactuation}{ Density actuation: symbols and quantities as in Figure \ref{f:volumeactuation}; on the vertical axis, the headway threshold is also shown declining to zero over the green extension period following the shaded lines in the background, which correspond to polynomial curves as in \req{e:dynamicthreshold} with fixed values of $\beta$. The maximum headway curve latches onto increasingly rapid decay curves with each arrival detected on other approaches, marked by the small triangles.}{width=\textwidth}

Although actuated controllers are mostly regarded as autonomous entities, it should be evident that phase duration limits and threshold function parameters associated with each approach can be finely tuned by a centralised system to deal with specific traffic scenarios.

Isolated actuated controllers are relatively undemanding from the infrastructural point of view, but the considerable drawback is that without junction coordination the flexibility in phase duration may come at a heavy cost in terms of arterial progression disruption.


\subsection{Automatic Plan Selection}
Plan selection systems, such as the \emph{Urban Traffic Control System} developed by the Federal Highway Administration, aim to ensure that the most suitable amongst a set of predetermined signal plans is enacted, on the basis of real time information about the traffic conditions.

Automatic plan selection is a straightforward enhancement for both isolated traffic lights and centralised traffic control systems, which could otherwise rely only on daily plan scheduling to use different plans tailored to specific traffic conditions. These plans can be developed offline using any of the techniques mentioned in Chapter \ref{c:basics}, with no concern for execution time or computational cost; stochastic search methods such as the one proposed in this work could well be used to devise plans for different times of day as well as response plans for specific events, with any performance objective of the planner's choosing.

Plan selection is typically performed by comparing real time detector readings with the conditions for which each plan was designed. Readings may be validated using historical data and otherwise filtered to protect the stability of the system against measurement errors and faults. The pre-processed input is then fed into an objective function that computes the degree of suitability for each plan.

Consider for example a bank of signalisation plans $s \in S$, each representing a \emph{solution} designed around a given traffic scenario — the generalisation applies at the network level just as well as for a single intersection, where the concepts of \emph{plan} and \emph{program} are equivalent.
Each scenario is represented by a snapshot of the traffic conditions: assume this to come in the form of flow and occupation values measured on a subset $\arcset^\oplus \subseteq \arcset$ of detector-equipped arcs of the network.

The core objective function of a plan selection method quantifies the degree of
\emph{coincidence} between the flow and occupancy values $\bar{\flow}_{a,s}$ and $\bar{o}_{a,s}$ associated with each of the pre-timed solutions with those measured on the corresponding network arcs in real time. A possible form for such a function is e.g.
\eq[,]{e:planselection}{
\omega_s = \sum_{a \in \arcset^\oplus} \alpha_a \cdot \left[
\beta_a^\flow \left(\flow_a - \bar{\flow}_{a,s} \right)^2 +
\beta_a^o \left(o_a - \bar{o}_{a,s} \right)^2
\right]}
where the current flow and occupation values $\flow$ and $o$ refer to each individual arc $a$, as do the location weights $\alpha_a$ (some locations may be strategically more important than others) and the measurement weights $\beta_a$ which reflect the relevance (or accuracy) of each reading at the given location.

Equation \req{e:planselection} can easily be extended to account for additional reading types.
The most suitable plan is the one that minimises the performance index $\omega_s$ , representing the divergence of the current traffic conditions from its signature traffic snapshot $(\mathbf{\bar{\flow}_s}, \mathbf{\bar{o}_s})$.
The system may further require the best candidate solution to beat the currently running
plan by more than a predefined threshold before confirming a plan change: a cautionary
measure called \emph{Anti-hunting} taken to avoid continuous switching between similar plans,
particularly in applications where a large number of plans are used to closely follow the
evolution of demand throughout the day.

Switching between different plans may momentarily disrupt corridor progression, therefore in
some cases a hybrid transition cycle is synthesised from the outgoing and incoming plans.
The above principles equally apply to single junctions, areas or entire networks, and require a
relatively low number of strategically placed detectors, making automated plan selection a
viable and cost-effective option for many applications.
 

\section{Real Time Signal Plan Generation}
Real time optimisers that perform plan generation are a class of proactive signal control systems that, based on current traffic conditions, seek to develop an optimal plan to apply in the immediate future, either from first principles or by continuous update of an existing pre-timed plan. While each plan plays out, the system gathers information to make the next.

This mode of operation is often referred to as rolling horizon, and in order for the system to respond effectively (i.e. to capture and react to rapid changes in traffic conditions) the rolling horizon time step should be reasonably short, which imposes austere constraints on the optimisation methods. Some real-time optimisers with a very short rolling horizon step update the signalisation plan at every cycle, so that their behaviour may appear indistinguishable from that of an actuated controller.

It is important however to understand the clear conceptual difference between the two: actuated controllers perform second-by-second decisions about the best action to perform instantly, while the systems considered in this section plan ahead, producing fully featured signal plans made of cycle times, offsets and green shares deemed optimal for dealing with the traffic conditions observed.

\subsection{Incremental Analytical Optimisation}
The most prominent member of this category is the \emph{Split Cycle and Offset Optimisation Technique} developed for research purposes in Glasgow, and first applied there in 1975 under the acronym SCOOT by which it is now popular all over the world, counting over a hundred active installations.

Continuous optimisation revolves around a centralised control unit which generates plans based on a real-time traffic snapshot gathered from detectors. The signalisation plans are continuously updated, with a frequency in the order of one to three cycle times, and may concern the entire network or \emph{regions} thereof which are expected to feature homogeneous traffic conditions.\\

One of the main advantages is that optimisation requires very little information about the network. All the system needs, for each approach to a controlled junction, is the following:
\begin{itemize}
\item \textbf{distance} from each detector (at least one is needed) to the stop line
\item \textbf{saturation flow} of the detector lane at the junction
\item total vehicle \textbf{storage capacity}
\item initial lost time and \textbf{clearance time} for the corresponding signal phase
\end{itemize}

The SCOOT optimisation method described in \citep{robertson1986research} is based on \emph{Cyclic Flow Profiles:} for each approach to a controlled junction these represent the continuously updated flow profile covering the span of a signal cycle with a resolution of 4 s, obtained from the readings gathered by sensors.
The centralised control unit integrates CFPs to estimate the number of vehicles arriving at the stop line while the signal is red, which combined with saturation flows yields the queue sizes and clearance times as pictured in Figure \ref{f:scoot}.

The system is therefore all the more effective if detectors are placed far from the stop line — possibly just downstream of the previous junction — to give as early a warning as possible of changes in the expected flow pattern. This also allows the system to detect significant spillback situations, triggering different operation modes aimed at gridlock avoidance.

It may be advisable to trade accuracy for early detection by overlooking minor side streets which may alter the flow rate and progression between two major intersections. However, best results are obtained with more sensors spaced out along each inbound arc.
With the flow conditions described by this simple traffic model, the optimiser proceeds to calculate cycle times, offsets and green shares based on explicit mathematical formulations (see Section \ref{s:signalsetting}). 

The fitness of the solution found is quantified by a global cost function built on a linear combination of delays and number of stops. The method runs as follows:
\begin{enumerate}
\item \textit{Cycle Time Computation:} each region of the network shares a single cycle time; its ideal value determined by an empirical formula similar to Webster's \ref{e:webstercycle} based on the saturation conditions at the critical (i.e. most saturated) intersection;

\item \textit{Green Share Optimisation:} once the cycle time is determined, green shares are updated at every intersection: as soon as it possesses relevant flow information, the optimiser decides whether to anticipate/delay each phase change by up to 4 s, depending on which alternative scores best according to an objective function (aiming to reduce the saturation level of the most saturated approach to the junction);

\item \textit{Offset Optimisation:} at every cycle, the central unit may shift the pre-timed offsets by up to 4s in either direction, if this leads to an improvement in an explicit objective function which accounts for the degree of synchronisation with the \emph{adjacent} junctions, possibly accounting for updated travel times of the relevant arcs.
\end{enumerate}

\fig{htbp}{PIX/scoot.png}{f:scoot}{
SCOOT Cyclic Flow Profiles and queue prediction: detector readings are used to update the flow profile, which is integrated to predict the queue forming at the downstream junction during the red phase. The information may prompt the system to anticipate or delay a phase change in order to accommodate the measured demand.}{width=0.8\textwidth}

This type of optimiser has the advantage of low modelling requirements and very fast computation times, combined with the ability to operate quite close to complete saturation —allegedly up to 90\% critical junction saturation.\\
Even with modest prediction capabilities and no full network model, it has proven capable of dealing reasonably well with moderate flow pattern alterations and unusual route choices such as might be caused by accidents or road works.

It does however rely heavily on the accuracy of detectors, which if insufficient may cause the performance of the system to decline rapidly: the modified timings are in fact set to degrade back to the pre-timed plan if sensor faults are detected.\\
The small adjustment step sizes are also chosen to increase the robustness of the system to detection faults: unfortunately, this goes to the detriment of its responsiveness, which has been pointed out as the main weakness of SCOOT.

\subsection{Linear Quadratic Optimal Control}
The \emph{Traffic Urban Control} system commonly referred to as TUC was developed in the scope of TABASCO (Telematics Applications in BAvaria SCotland and Others), a late '90s European project aimed at demonstrating the applicability of advanced transport telematics as innovative solutions for traffic management. Initially conceived for green split optimisation, it was extended to deal with cycle and offsets as well, and later enabled to perform on-the-fly Public Transport prioritisation.

It therefore constitutes a direct alternative to the SCOOT system mentioned in the previous section, and was designed to build upon the latter’s ease of applicability while addressing its main issues: most notably its slow response to rapid traffic variations (due to the incremental correction approach), and scarce effectiveness under high saturation conditions. The former was made unnecessary by the verified robustness and stability of the system, while a stronger interdependence of measurements and signal settings across the entire network helped counteract the tendency shown by more localised control policies to accelerate the onset of saturation by blindly favouring high flows.

The system inputs are the average numbers of vehicles on network links (which may be estimated from occupancy readings if video detection is not possible) and public transport information, at least accurate enough to detect the \emph{presence} of public vehicles on a given link. Cycle and offset optimisation are carried out independently and in much the same way as it was described in the previous section.

What characterises the TUC strategy however is its approach to \emph{green split optimisation}, based on a Store-and-Forward traffic model \citep{aboudolas2009store} and simple control theory.
These are combined to formulate the control problem as a Linear Quadratic optimisation, as illustrated in detail in \citep{diakaki2002multivariable} and summarised here.

The instantaneous network state is represented solely by the number of vehicles on each link. The discrete-time evolution rule of the network dynamic system encapsulates its dependency on the decision variables and on previous instantaneous states, and in matrix form may be written simply as
\eq[,]{e:tuc1}{
\mathbf{n_{a,t+1}} = \mathbf{A n_{a,t}} + \mathbf{B \Delta \gamma_{p,t}}
}
where $\mathbf{A n_{a,t}}$ is the vector of states containing the number of vehicles on each link and $\mathbf{\Delta \gamma_{p,t}}$ is the
vector of variations in green share applied to each signal phase, with respect to a baseline
signal plan assumed to lead to steady state queues under non-saturating conditions.

$\mathbf{A}$ and $\mathbf{B}$ are the state and input matrices: they respectively encapsulate the network topology and the expected impact of signalisation (based on signal staging, turning rates,
saturation flows) on the movements of traffic volumes across time intervals.

It should be noted that the \emph{expected} demand is taken into no account: this is reasonable as
TUC aims to react to the \emph{manifest} impact of disturbances on the controlled network rather
than to their forecast consequences.
At the core of this approach lies a simple gain matrix — introduced in Equation \req{e:tuc3} —
rather than accurate modelling of physical phenomena and constraints: its calculation and
calibration however are most computationally demanding processes.

In order to minimise the risk of oversaturation and queue spillback on all network links,
the chosen strategy is to attempt to balance the link relative occupancies (with respect to each
link’s jam storage capacity), as expressed by the following quadratic criterion:

\eq[,]{e:tuc2}{
\omega^{\scriptscriptstyle{TUC}} = \frac{1}{2} \sum_{t=0}^{\infty}
\| \mathbf{Q \, n_{a,t}} \|^2 + \| \mathbf{R \, \Delta \gamma_{y,t}} \|^2
}
where $\mathbf{Q}$ and $\mathbf{R}$ are non-negative definite diagonal matrices of weights, so that the cost function is compatible with the standard form of a Linear Quadratic Cost.

Matrix $\mathbf{Q}$ contains the inverse storage capacities of links, so that the first term of the sum drives the relative occupancy balancing, while the second term favours smooth changes in the control variables, influencing the magnitude of control reactions through appropriate scaling factors contained in matrix $\mathbf{R}$.
The infinite time horizon of the sum reflects the necessity to obtain a time-invariant feedback control law in accordance with LQ optimisation theory.

The LQ feedback control law is then obtained by minimisation of the performance
criterion \req{e:tuc2} subject to \req{e:tuc1}: calculation of the control matrix $\mathbf{L}$ is straightforward, but can only be performed offline by solving the infinite-horizon \emph{Discrete-time Algebraic Riccati Equation} from the network topology and objective function weights described by the matrices $\mathbf{A}$, $\mathbf{B}$, $\mathbf{Q}$, and $\mathbf{R}$. These must be computed and calibrated individually for the specific network topology, capacities, signal staging etc, by simulation or other optimisation methods: this is a lengthy and demanding task to be performed as part of the system setup.

However, after finding the stabilising solution $\mathbf{L}$ to the dynamical system expressed by
the DARE, things get much simpler, with the control law taking the standard form
\eq{e:tuc3}{\gamma_{p,t} = \bar{\gamma}_{p} - \mathbf{L \, n_{a,t}}}
where $\bar{\gamma}_{p}$ is the vector of baseline green shares. Optimal modifications to the green times are linearly dependent on the current network state vector of link occupancy measurements through the matrix $\mathbf{L}$, which provides both the discharge and gating functionalities: intuitively, as the occupancy of a link increases, so does the green share that favours its outflow, while upstream arcs experience a reduction in green time to avoid its oversaturation.\\
These effects can be accentuated or mitigated by weighing elements of the state vector
according to specific rules, e.g. to prioritise desaturation of certain links as they approach
critical saturation levels.

A simple form of \emph{public transport prioritisation} can be integrated into the
application of the control equation \req{e:tuc3}, by further weighting link occupancy values in
function of the number of public transport vehicles detected on them.

Since control constraints such as green time upper and lower bounds cannot be directly
accounted for by the LQ methodology, the green shares output by the regulator are further
processed on the fly by a simple optimisation algorithm that, in linear time, finds the set of
feasible green times that least deviate from the optimum. 

Although several software packages are available for solving the DARE for this standard
LQ control problem using a variety of well documented methods, the calculation of an
effective control matrix remains a time consuming task, particularly for large networks, and
must be performed anew every time the controlled network is modified or extended.\\
This lack of flexibility represents the main drawback of the approach just presented,
although it has been proven that reasonable variations of traffic parameters such as turning
rates and link saturation flows have little effect on the control matrix.\\
On the other hand, the real-time operation of the TUC control strategy only consists of
the solution of the simple matrix equation \req{e:tuc3} followed by the application of green time
constraints, which are both extremely fast and undemanding operations, making the quadratic
regulator a particularly suitable approach for real time applications. Furthermore, the
feedback controller is perfectly capable of responding appropriately to very specific traffic
anomalies such as accidents or roadworks, as confirmed by both simulation and empirical
data gathered from real world installations.

Since the first TUC installation in Glasgow, further applications of optimum control
theory to the signal setting problem, including open-loop Quadratic Programming and
Nonlinear Optimum Control based on the same store-and-forward traffic paradigm, have been
developed and investigated. These aim to improve upon the performance of the simple
feedback controller by accounting for more detailed network dynamics, factoring in time
varying demand, or allowing for a larger and more effective set of decision variables:
encouraging results presented in \citep{diakaki2003extensions} suggest that despite an increased real-time computation complexity, these may be considered strong competitors and potential successors to the Linear Quadratic TUC approach.

\subsection{Traffic Gating}
\newcommand{\tts}[1]{\omega_{#1}^{\scriptscriptstyle{TTS}}}
\newcommand{\ttd}[1]{\omega_{#1}^{\scriptscriptstyle{TTD}}}

Feedback Traffic Gating as described by \cite{keyvan2012congestion} is a form of actuated signal control aiming to prevent oversaturation of critical portions of the network by holding back the incoming traffic flows — using deliberately exaggerated red phases — rather than attempting to deal with the flows already trapped in a congested area. 

In these respects, it constitutes a simple yet innovative method to induce more efficient utilisation of the existing infrastructure, and an answer to the patent performance degradation that currently feasible real-time optimisation solutions face under saturated conditions; it is therefore also closely related to the object of this study.

\fig{htbp}{PIX/gating_intro_simple.png}{f:gating}{
Traffic Gating: a cordon of gating junctions holds back traffic attempting to access the \emph{Protected Network}. If it is not possible to implement gating at one or more cordon junctions (see $j$ = 2), these may allow some disturbance flows to sneak past the feedback controller. Conversely, part of the gated flow from a cordon junction may not in fact be bound for the PN (see $j$ = 4). The system must account for the delay between control application and effect (due to the physical distance between the gating junctions and the PN, see $j$ = 3) and incomplete or uneven detector placement in the protected network.
}{width=\textwidth}

Based on the general principle that even from the users’ point of view there is no advantage to getting close to one’s destination sooner, only to be stuck in traffic for longer, the system delays incoming vehicles in order to keep the controlled network near to but \emph{below} its saturation occupancy level, which can be monitored effectively even with a small number of detectors \citep{keyvan2013urban}: this was proven the most effective strategy to maximise \emph{network throughput}, which constitutes a good measure of how efficiently the network is being used.

A feedback controller is used to ensure that the only vehicles pre-emptively delayed are those which not only would, on average, be delayed anyway further down their path, but would \emph{critically} increase congestion — causing themselves  as well as others greater delays were they to access the critical region.

\subsection{Network Fundamental Diagram Formulation}
Feedback Traffic Gating revolves around the concept of Network Fundamental Diagram
introduced in \citep{keyvan2012congestion}, profiling throughput as a function of occupancy, as seen in Figure \ref{f:nfd} where the axes of the sample NFD correspond to \emph{Total Time Spent} (in vehicle-hours per hour) and \emph{Total Travelled Distance} (in vehicle kilometres per hour) cumulatively by all users hourly. 

Such relationship may be obtained empirically from observation of the
area of interest, and allows to identify with certainty the optimal operation point for the
feedback controller to suit the behaviour of a specific network.

\fig{htbp}{PIX/NFD.png}{f:nfd}{
An experimental Network Fundamental Diagram: a polynomial fit of the \emph{TTD} curve is obtained from flow and occupancy measurements, identifying an optimum operation point on the \emph{TTS} axis. As long as $\tts{\arcset^\oplus}$ is kept within the optimum operation range, the number of vehicles in the Protected Network is expected to maximise the infrastructure efficiency, resulting in shorter travel times for all users. The dynamics of the system during network loading and unloading can be expected to differ as flows tend to be slower during the relaxation of a more congested state.
}{width=0.65\textwidth}

An operational NFD is derived from real or simulated occupancy measurements a o
taken on a set of detector equipped arcs $\arcset^\oplus \subseteq \arcset$ at discrete time intervals $t$ corresponding to signal cycles.
Occupancy is converted into an estimate $n_{a,t}$ of the number of vehicles on each
arc during the $t^{th}$ signal cycle, given by

\eq{e:gating1}{
n_{a,t} = \frac{\length_a \cdot o_{a,t}}{100 \, \length_{veh}}
}

where $\length_a$ is the length of link $a$, $o_{a,t}$ its occupancy (given as time percentage) during the interval, and $\length_{veh}$ the average vehicle length. Hence, the relevant quantities are obtained by summing over the measurement arcs:

\eq[;]{e:gating2}{
\tts{t} = \sum_{a \in \arcset^\oplus} \frac{n_{a,t} \cdot \cycle{\junset}}{\cycle{\junset}} =
\sum_{a \in \arcset^\oplus} n_{a,t}
}

\eq[.]{e:gating3}{
\ttd{t} = \sum_{a \in \arcset^\oplus} \frac{\flow_{a,t} \cdot \length_a \cdot \cycle{\junset}}{\cycle{\junset}} =
\sum_{a \in \arcset^\oplus} \flow_{a,t} \cdot \length_a}

The values thus obtained are sufficiently precise for the purpose of traffic gating, especially if
detectors are located around the arc midpoints. Although a high number of detector links
(ideally $\arcset^\oplus = \arcset$) yields a more accurate NFL, \citep{keyvan2013urban} proves that fully functional results can be obtained also from a \emph{reduced} NFL in more likely scenarios where only a costeffective subset of links has detection capabilities, such as would be sufficient for ordinary traffic monitoring, plan-selection schemes, or actuated signal control applications, as portrayed e.g. in Figure \ref{f:gating}.

\subsubsection{Feedback Controller Design}
\newcommand{\qg}{\flow^{g}}
\newcommand{\qe}{\flow^{\epsilon}}
\newcommand{\qin}{\flow^{\scriptscriptstyle{IN}}}
\newcommand{\qout}{\flow^{\scriptscriptstyle{OUT}}}
\newcommand{\roin}{\rho^{\scriptscriptstyle{IN}}}
\newcommand{\npn}{n_{\scriptscriptstyle{PN}}}

The gating control problem is to regulate the \emph{TTS} in the Protected Network via appropriate manipulation of gated inflows, so as to maintain the \emph{TTD} around its optimal maximum value. The task can be accomplished as summarised in Figure \ref{f:gating2} based solely on real-time measurements via a simple and robust feedback regulator, taking advantage of the basic system dynamics described by the NFD.

\fig{htbp}{PIX/gating_controller.png}{f:gating2}{
Gating Feedback Controller and Protected Network Dynamic Plant.}{width=\textwidth}

The controlled input to the PN system is the gated flow $\qg$, and the main disturbance
the uncontrolled inflow $\qe$. Referring to Figure \ref{f:gating2}, the inflow $\qin$ in continuous time is 
\eq{e:fc1}{
\qin_t = \roin \cdot \qg \left( t- t^{access} \right)
}
where $\roin$ is the portion of gated flow entering the PN, and $t^{access}$ the time it takes for vehicles to reach the PN from gating junctions not directly located on its boundary.

Consider the total number $\npn$ d of vehicles in the PN: its rate of change is determined
form vehicle conservation, which reads
\eq[.]{e:fc2}{
\dot{\npn} = \qin + \qe - \qout
}

However, $\tts{\scriptscriptstyle{PN}} = \npn $ only if all PN links are monitored, which is not generally the case.\\
Realistically, the \emph{TTS} is smaller than the true number of vehicles by some factor $\rho^{\epsilon 1} \leqslant 1$.
Allowing for an additional measurement error $\flow^{\epsilon 1}$ the \emph{TTS} value to be used in the NFD is

\eq[.]{e:fc3}{
\tts{\scriptscriptstyle{PN}} = \rho^{\epsilon 1} \cdot \npn + \flow^{\epsilon 1}
}

and finally, if $nfd( \tts{\scriptscriptstyle{PN}} )$ is a nonlinear best fit of the NFD data (see Figure \ref{f:nfd}) and $\flow^{\epsilon 2}$ the error due to the data scatter, the resulting \emph{TTD} is

\eq{e:fc4}{
\ttd{\scriptscriptstyle{PN}} = nfd \left( \tts{\scriptscriptstyle{PN}} \right) + \flow^{\epsilon 2}
}

which as seen in Figure \ref{f:gating2} is proportional to the network outflow $\qout$ aside for a scaling factor $\rho^{\epsilon 2}$ analogous to $\rho^{\epsilon 1}$, yielding a time delayed nonlinear first-order model between the initial $\qg$ and the resulting \emph{TTS} which can be linearised around the optimum steady state.

The following proportional-integral controller is then well suited to handle the gated flows:

\eq{e:fc5}{
\qin_t = \qin_{t-1} - K^P \, \left( \tts{t-1} - \tts{t-2} \right) + 
K^I \, \left( \bar{\omega}^{\scriptscriptstyle{TTS}}  - \tts{t-1} \right)
}

where $K^{I}$ and $K^{P}$ are the integral and proportional gains to be fine tuned.\\
The flow values thus determined have to be shared amongst all gated junctions, after accounting for monitored or estimated disturbance flows, and subjected to minimum/maximum green time constraints.

The resulting system is largely robust to measurement errors, low signal timing resolution,
and fluctuations in demand. It may be activated at specific times or as the traffic conditions
approach a critical state, and requires virtually no additional infrastructure with respect to an
ordinary plan-selection centralised signal setting system. Provided that appropriate gating
locations can be found, where gate-delayed flows do not risk compromising the mobility of
vehicles not bound for the PN, the principles just illustrated will undoubtedly form the core of
future sustainable approaches to relieve urban congestion by delaying or avoiding the extreme
traffic conditions that frustrate most currently available signal optimisation techniques.











