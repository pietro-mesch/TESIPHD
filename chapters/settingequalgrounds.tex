\section{Setting equal grounds for comparison}
\todo{How to put Balance and TRE on equal grounds for a significant performance comparison.}

Considering the Balance requirements and settings illustrated in \ref{s:balreq} and \ref{s:balset} respectively, it is possible to try and put Balance in operation conditions that are as close as possible to those of the TRE based Bandwidth optimiser.

This requires restricting the decision space of Balance while providing it the maximum possible detection capabilities, and further aligning the cost functions to specifically compare the efficacy of the algorithms while factoring out all expected sources of discrepancy.

\subsubsection{Model setup}
\todo{balance detectors}

Vissim acceleration.

Vissim stochasticity removed.

Problems with connectors!
At the start of connectors TRE is ok.
At the very start of the entry links TRE is missing some but they're accounted for on the relevant connector.

\begin{tabular}{|c|c|c|c|}
\hline 
ZONE & VISUM & VISSIM & TRE \\ 
\hline 
1 & 672.5 & 665 & 635.1 +36.3\\ 
2 & 190 & 190 & 179.4 +10.3\\ 
3 & 180 & 180 & 170 +9.7\\ 
4 & 41.67 & 41 & 39.6 +2.3\\ 
5 & 91.67 & 91 & 86.6 +5.0\\ 
6 & 30 & 30 & 28.3 +1.6\\ 
7 & 30 & 30 & 28.3 +1.6\\ 
\hline 
\end{tabular} 

Discrepancy can be minimised by setting zero-length connectors, but not really eliminated.


Vissim flows seem capped at 1800 veh/hr regardless of the link capacity in visum.


\subsubsection{Balance Decision Space}
\todo{limiting balance to junction offsets only: Balance lavora stage based, cambiando gli istanti di inizio degli interstage.
Per forzare il comportamento "solo offset" bisogna bloccare la durata degli stage, e lasciare libertà totale a tutti gli interstage.}


\subsubsection{Cost Function}
\todo{Weighting out side approaches from Balance}

\todo{Reproducing Balance PI in TRE}