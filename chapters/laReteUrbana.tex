\chapter{Signalisation of Urban Networks}
\todo{Chapter intro and overview: elements of a signal plan and how they should be designed to serve the purpose.}
\section{Anatomy of a Signal Plan} \label{s:anatomy}
The following section briefly illustrates the main features of a signal plan devised for urban traffic regulation. This term encompasses all timings and schedules behind the delicate clockwork of traffic signals, from the elements that constitute a single signal program at one of the many junctions of the network, to the succession of network-wide program changes designed to meet the daily evolution of traffic demand and the propagation of vehicle flows.
The features presented in this section fully define what is commonly called a pre-timed plan, and as such do not describe any real-time actuation or decision making logic.
They are themselves, however, the decision variables of most optimisation methods and adaptive strategies, and it is crucial to understand their significance in order to appreciate the diversity of setting and control approaches illustrated in more detail throughout this chapter.

\fig{htbp}{PIX/signalplan.png}{f:plan}{
Elements of a network-wide signal plan: a daily schedule specifies the signal programs running at each intersection. The sequence and duration of signal phases repeats over the course of every signal cycle as specified by the different signal programs, administering junction capacity amongst the expected traffic flows. During each phase, a set of compatible manoeuvres is allowed through while the others remain closed.}{width=0.7\textwidth}

\subsubsection{Signal Phases}
Traffic signals exist mainly to separate conflicting traffic flows competing for the right of way at a road intersection. The natural way of doing so is to bundle compatible (e.g. non-secant) manoeuvres which may be safely performed simultaneously into signal phases, so that the corresponding flows may be allowed through the junction in turn.
Phases are the fundamental blocks of a signal program, and are usually repeated in the same order at every signal cycle, although some signalisation systems provide phase skipping, usually as part of their public transport prioritisation strategy.
Manoeuvres may pertain to different modes of transport, meaning that cars, trams and
pedestrians are taken into joint consideration and can be given the right of way during the
same signal phase.

Consider a junction, i.e. a network node $j \in \nodset$ where it is possible to perform a given set of manoeuvres $\manset_j$ .
The generic manoeuvre $y \in \manset_j$ may be:
\begin{itemize}
\item a turn, from an arc $ a \in \arcset_j^- $ of the node's backward star, to a forward star arc $b \in \arcset_j^+$ ;
\item a tram crossing or similar transport system specific operation;
\item a pedestrian crossing affecting one or more arcs either entering or leaving the junction.
\end{itemize}

In order to present a straightforward definition of \emph{manoeuvres} in relation to junction layout and signalisation, the focus will henceforth be on the movement of private vehicles only, unless otherwise specified.
It shall be clear that the principles of manoeuvre compatibility illustrated in this manner in Figure \ref{f:phasing} may be easily generalised to different and etherogeneous modes of transport, such as public transport, pedestrians and bicycles.

\fig{htbp}{PIX/phasing.png}{f:phasing}{
Manoeuvres at an intersection, conflict areas and possible phasing options: option A avoids
direct conflicts between Eastbound (E-) and Westbound (W-) manoeuvres, as would be desirable if high volumes were expected along that direction;
option B favours a lower number of phase changes (less time lost) assuming flows to be such that left turning vehicles have space to wait at the middle of the intersection, until the oncoming through flow decreases enough to let them cross.}{width=\textwidth}

Given the layout of a junction $j$, different manoeuvres may or may not
be safe to perform simultaneously, as exemplified in Figure \ref{f:phasing}.
This information, which may well depend on the flow conditions, is easily represented by a square Boolean matrix where rows and columns correspond to each manoeuvre and elements comply with the following rule:

\eq[.]{eq:phasing}{
\delta_{yz} = \left\lbrace 
\begin{array}{ll}
1 & \text{if $y$ and $z$ are compatible} \\
0 & \mathrm{otherwise}
\end{array}
\right.
\forall y,z \in \manset_j
}

Each possible subset of manoeuvres $p \subseteq \manset_j$ potentially identifies a \emph{signal phase}.
A viable set of phases $\phaset_j$ for the junction however must belong to the space of \emph{feasible} signal phases, i.e. all possible sets of manoeuvres contained in the power set $ \wp \left( \manset_j \right)$ whose elements are mutually compatible according to \req{eq:phasing}.
The union of all phases must also include every available manoeuvre at least once.

Formally, $\phaset_j$ must therefore comply with the following properties:
\eq[.]{eq:phaseset}{
\phaset_j = 
\left\lbrace 
p \in \wp \left( \manset_j \right) : \prod_{y \in p} \prod_{z \in p} \delta_{yz} = 1
\right\rbrace
\quad , \quad
\bigcup_{p \in \phaset} p = \manset_j
}

Clearly, the power set $\wp \left( \manset_j \right)$ contains sets of manoeuvres that, although compatible and technically feasible, make little practical sense.
The selection of an optimal set of phases $\s{P}_j$ satisfying relation \req{eq:phaseset} with respect to a specific objective (e.g. minimum total delay for given demand flows) is a combinatorial bi-level problem, usually solved through a \emph{what-if} approach in which the selection of a good set of phases remains largely a traffic engineer's task.

Conceptually, the determination of signal phases is thus driven by the interactions
between manoeuvres. From a practical point of view, however, administration of the right of
way by means of traffic signals cannot transcend the junction layout.
For example, it is only possible to separate manoeuvres into different phases if each has a dedicated lane that allows vehicles to queue for it without hindering traffic that is headed elsewhere. In fact, as everyday experience testifies, traffic signals do not allow or prohibit
manoeuvres directly, but rather regulate vehicle egress from lanes (or lane groups) dedicated
to specific sets of manoeuvres.

Each lane or group of adjacent lanes $a$ sharing the same manoeuvre set $\manset_a \in \manset_j$ can be conceptually assimilated into a \emph{lane group}: a single independent arc $a \in \arcset_j^-$
of the node backward star.
Let $\arcset_p$ be the set of lane groups which are given the green light during signal phase $p$,
and $\manset_a$ the manoeuvres that can be performed from lane group $a$.
The set of manoeuvres allowed during phase $p$ is therefore
\eq[.]{eq:lights}{
p = \bigcup_{a \in \arcset_p} \manset_a
}
The set of manoeuvres $\manset_a$ specific to each lane group $a$ is relevant for the determination
of the arc effective outflow capacity, which may be affected by partial conflicts with other
manoeuvres allowed during the same phase. The HCM (2010) manual \ref{???} presents practical
methods for quantifying such effects.

\subsubsection{Signal Programs}
A signal program contains the state switching times for all signals at a given junction.
For signal planning and optimisation, it is practical to view the program as a succession
of signal phases with specific durations, as portrayed in Figure \ref{f:plan}: during each phase a set of arcs are open, allowing users to carry out the corresponding manoeuvres, while the others arcs remain closed and accumulate queues.

A program for junction $j$ consists therefore of a cyclic set of instructions spanning a period called \emph{cycle time} $\cycle{j}$: given a phase set $\phaset_j$, these specify the start and end of each signal phase with respect to the beginning of the signal cycle.

Transitions between subsequent phases are usually enacted via pre-timed signal state change sequences that handle the closure of a set of lane groups before opening the next.

\subsubsection{Daily Schedule}
It is common practice to tailor several signal programs to the traffic conditions normally
observed at different times of the day, in order to meet each scenario with the best possible
allocation of resources. The daily schedule defines the sequence of programs that each
junction will run over the course of the day.

\subsubsection{Cycle Time}
The cycle time $\cycle{j}$ is the \emph{period} of the signal program, i.e. the time lapse between two occurrences of the same signal phase at a given junction.
It affects the average delay and the level of saturation at which the intersection may operate. In general, longer cycle times imply larger average delays, but increase the total throughput, which may be necessary to deal with high demand flows by attenuating the effects of the time lost in signal phase changes.

\subsubsection{Effective Green Shares}
The nominal duration of each phase $t_p$ is seldom exploited by demand flows at the full
capacity of the corresponding arcs: even assuming that vehicles are not held back by
downstream congestion, it is necessary to account for some transient phenomena affecting the
performance of a junction.

As the signals turn green at the beginning of each phase, some time is lost before the
queuing vehicles start moving, and some more passes before the flow through the stop line reaches the arc capacity.
On the other hand, if a lane group remains open during two subsequent phases such effects will be smaller, in proportion.
After taking into account all delays and extensions, the portion of cycle time during
which a given lane group a may allow traffic onto the junction at full capacity is referred to as
its \emph{effective green share}.
The absolute and relative durations of effective green experienced by lane group $a$ during phase $p$ are denoted respectively as:

\eq[.]{eq:gshr}{
\gren{a,p} \in \left[ 0, t_p \right]
\qquad \text{and} \qquad
\gshr{a,p} = \frac{\gren{a,p}}{\cycle{j}}
}

It is not uncommon to have a lane group open during more than one phase: typically, an
approach experiencing high traffic volumes is given the right of way over two or more
consecutive phases without incurring further lost time in the phase change.

The effective green of each arc $a$ is then calculated from the total effective green time it
gathers over all relevant phases:
\eq[.]{eq:gshrtot}{
\gren{a} = \sum_{p \in \s{P}_j} \gren{a,p} \quad \text{with} \:
\left\lbrace \begin{array}{ll}
0 < \gren{a,p} \leq t_p 	&	\text{if} \, a \in \arcset_p	\\
\gren{a,p} = 0				&	\text{otherwise}		\\
\end{array} \right.
\quad \text{and} \quad
\gshr{a} = \frac{\gren{a}}{\cycle{j}}
}

\subsubsection{Signal Offset}
When multiple signals are involved, it is important to consider that vehicles that cross a signalised junction become packed into \emph{platoons}, which will eventually reach yet another signal-controlled stop line: adjusting the relative timing of adjacent junctions so that platoons meet a green light greatly affects the average delay incurred by the user. 

Synchronisation issues are addressed by defining a global time reference, with all junctions sharing the same cycle time or integer fractions thereof.
Each junction may then have all of its phase switching times anticipated or delayed in order to operate in concert with the neighbouring ones.
The amount of time $\toffs{j}$, by which the beginning of a cycle at one junction $j$ lags or leads the global reference instant, is referred to as a positive or negative \emph{offset}, respectively.

\section{Signal Setting}
Long before microprocessors and sensors made adaptive real-time traffic control an everyday reality, the notion of signal plan optimisation identified with a range of techniques for designing good signal plans based on historical demand flows, which will henceforth be referred to as \emph{offline} signal setting.

It is worth noting that such methods are not only still used for planning, but lie at the core of several adaptive signal setting approaches: once a \emph{signal setting policy} is chosen to determine the best signalisation parameters for given traffic conditions, it makes little difference from the methodological point of view whether the input variables are determined from historical data or fed in real-time by sensors. 

Naturally, the notion of offline planning does not imply that the dynamic interaction between signal setting and driver behaviour can be disregarded: for example, the assumption often made that route choices are fixed and unaffected by signal settings has warranted the formulation of planning strategies which have proven quite patently inadequate in the real world, as first discussed in Dickson (1981). While optimisation of a single junction for given flows may be a relatively simple problem with an analytical solution, devising a plan for an entire network is an entirely different task.

This section introduces the fundamentals of network signalisation design, describing the methods commonly used to determine the foremost features of a signal program, including cycle time, offsets and green share allocation.

\subsection{Performance of Isolated Signalised Junctions} \label{s:performance}
\todo{only a more specific case of junction}

The concept of \emph{performance} of a signalised junction may be defined in several ways, but in general terms it represents a gauge of the interaction between supply and demand with respect to a choice of metrics. As such, it depends on the junction physical layout, on the distribution of vehicle arrivals in time and on the signal that regulates their departure times. 

Several flow models were introduced in the scientific literature to reproduce arrival and departure phenomena.
For all signal planning purposes, traffic flow is usually assimilated to a fluid stream according to the \emph{macroscopic} paradigm, which differs substantially from the microscopic approach where the trajectory of each single vehicle is explicitly considered.

More specifically, vehicle \emph{departures} from a stop line are modelled as a uniform flow. If the \emph{arrival} flows are sufficiently lower than capacity, their inherent random component can be neglected and they are also considered deterministic.
Conversely, if stochasticity of arrival flows is significant, as it occurs when they approach the relevant arc capacity, or are very low, a random component is added to the simple deterministic model as in \todo{ref Webster, (1956)}.
This section will present the basic relationships between signal timing variables and junction performance with reference to the simple deterministic model.

\subsubsection*{Queues and Queue Clearance}
Consider a single arc (lane group) $a \in \bstar{j}$ entering a signalised junction $j \in \nodset$, with a constant demand flow of vehicles $\flow_a$ arriving over the entire cycle. The flow can only be discharged onto the junction during the effective green time, at the constant saturation flow rate $\satflow_a$ given by the arc capacity and possibly degraded due to conflicts with other arc flows. The \emph{flow ratio} between demand and saturation is denoted as:
\eq[.]{e:flowratio}{
\flowratio_a = \frac{\flow_a}{\satflow_a}
}
During the rest of the cycle, the departure rate is zero and vehicles have to stop, forming a \emph{queue}, which has to be discharged during the next green phase if it is not to grow indefinitely. 

The saturation flow $\satflow_a$ must therefore be sufficient to serve the queue accumulated over the red phase, which has duration $\cycle{j}-\gren{a}$, in addition to the flow of vehicles that keep arriving during the green phase $\gren{a}$.

This relationship is illustrated in figure \ref{f:queues} and may be formalised by considering the following expression for the \emph{queue clearance time} in terms of the signal timing and flows just described:
\eq[.]{e:tclear}{
\tclear{a} = 
\frac{\flow_a \left( \cycle{j} - \gren{a}\right)}{\satflow_a - \flow_a} =
\frac{\flowratio_a (1-\gshr{a})}{1-\flowratio_a} \: \cycle{j}
\quad , \quad
\forall a \in \bstar{j}
}

\subsubsection*{Vehicle Stops}
In this context, it makes sense to assume that vehicles will stop if they reach the stop line during the red phase or if they have to join the back of a queue that has yet to be fully discharged, although this is a slightly conservative approximation as the back of the queue might not be standing still during the green phase.

The number of vehicles that end up stopping (or significantly slowing down) during every signal cycle can therefore be expressed as
\eq{e:queuestops}{
\nveh{a} = \flow_a \left(\cycle{j} - \gren{a} + \tclear{a} \right) = 
\satflow_a \: \tclear{a}
}
where the right-hand side equality is justified simply by the definition of clearance time $\tclear{}$ given by equation \req{e:tclear} under the assumption that standing vehicles will discharge onto the junction at the maximum possible flow rate during the effective green phase. 

This in turn leads to the theoretical definition of the \emph{stop ratio}, an essential metric indicating what fraction of the total flow of vehicles will have to stop at the junction:
\eq[,]{e:stopratio}{
\stopratio{a} = \frac{\satflow_a \tclear{a}}{\flow_a \cycle{j}} = 
\frac{1-\gshr{a}}{\-\flowratio_a}
}
which is proportional to the red share of the cycle time and increases as the arrival rate approaches the discharge capacity.
Quite obviously for values of $\flowratio_a \geq 1$, but also if $\gshr{a} < \flowratio_a$ queues cannot be fully discharged at every cycle, and all vehicles end up stopping: in this case, the queue can grow indefinitely.

\fig{htbp}{pix/queues.jpg}{f:queues}{
Geometric determination of stopped vehicles and queue clearance for one approach given the relevant demand flow, saturation flow, cycle and green time.
The grey triangle between the arrival cumulative, the departure cumulative and the horizontal axis covers the number of vehicles queuing at
any given moment. 
Notice that the number of standing vehicles $\nveh{a}^standing$ at the beginning of the effective green does not account for all vehicles that need to stop $\nveh{a}^stop$ according to the approximation given by equation \eqref{e:queuestops}.
}{width=0.8\textwidth}

\subsubsection*{Average Delay}
Assuming constant arrival and departure rates, the total delay experienced at each cycle by all users from a given approach a corresponds to the integral over time of the queue size (the area of the greyed out triangle in Figure \ref{f:queues}), whence the average delay $\avgdelay{a}$ per vehicle is found to be
\eq[,]{e:avgdelay}{
\avgdelay{a} = 
\frac{\left( \cycle{j} - \gren{a} \right) \left( \satflow_a \: \tclear{a} \right)}
{2 \: \left( \flow_a \: \cycle{j} \right)} =
\frac{\left( \cycle{j} - \gren{a} \right)^ 2 }
{2 \: \left( 1 - \flowratio_a \right) \: \cycle{j}}
}
using \req{e:tclear} for the queue clearance time $\tclear{a}$.

Clearly, the above equation \req{e:avgdelay} assumes no standing queues at the end of a cycle. More complex delay functions can be obtained by considering stochastic fluctuations of arrival flows \todo{ref (Webster, 1958)}. 
Flows exceeding the arc capacity require the introduction of either simulation models or empirical adaptations of analytical models, such as the coordinate transformation method introduced by  \todo{ref Kimber and Hollis (1979)} and later adopted by the
popular  \todo{ref HCM traffic manual (2010)}.

\subsubsection*{Critical Flow Ratio and Saturation}
The saturation flow characterising each lane group depends on various factors, such as
\begin{itemize}
\item total road width,
\item visibility,
\item conflicts with other manoeuvres served during the same phase,
\item presence of dedicated turn bays to alleviate such conflicts.
\end{itemize} 
Conflicts are particularly relevant to left turns, or turns encroaching a pedestrian crossing: scrupulous phase planning can minimise the number and entity of such conflicts.

The flow ratio $\phi_a$ quantifies the expected demand on a given lane group $a$ in relation to its \emph{nominal} saturation capacity.
The saturation level $\saturation_a$ is determined by the ratio of demand flow to its \emph{outflow capacity}, which is further limited by the signal, inasmuch as each arc can only be open for a limited share of the available green time:
\eq[.]{e:saturation}{
\saturation_a = \frac{\flow_a}{\gshr{a} \, \satflow_a} = 
\frac{\flowratio_a}{\gshr{a}}
}
For values of $\gshr{a} < \flowratio_a$ the saturation level is above 100 \% and the flow cannot be served, leading to queues that grow indefinitely until demand drops.

When multiple lane groups are to be open simultaneously during phase $p$, the \emph{critical flow ratio} $\flowratio_p$ is given by the approach which is relying most heavily on the phase in question.
The concept is formalised in equation \req{e:critflowrate} by scaling the flow ratio of each approach in proportion to the share of its green time represented by the current phase.

In other words, in searching for the maximum flow ratio, only the share of flow that each lane group must serve during the specific phase is considered:
\eq[,]{e:critflowrate}{
\flowratio_p = \max{\left\lbrace \flowratio_a \frac{\gshr{a,p}}{\gshr{a}} 
\; | \: a \in \arcset_p \right\rbrace}
}
whence conversely the \emph{critical lane group} of phase $p$ is also identified as
\eq[.]{e:critlanegroup}{
\arcset_p^* = \left\lbrace a \in \arcset_p \; | \: \flowratio_p = 
\flowratio_a \frac{\gshr{a,p}}{\gshr{a}} \right\rbrace
}

The \emph{critical saturation} of signal phase $p$ is obtained by applying \req{e:saturation} to its critical lane group:
\eq[,]{e:critsaturation}{
\saturation_p = \frac{\flow_p}{\gshr{\arcset_p^*}}
}
noting that in the particular case where each lane group is only open during a single phase, critical saturation occurs on the one registering the highest flow ratio.

Since different lane groups may experience different effective green shares, should be calculated using the effective green experienced by the same lane group during that phase, which is practically considered the \emph{phase effective green}:
\eq[.]{e:effectivephasegreen}{
\gren{p} = \gren{\arcset_p^* , p}
}

Finally, the total \emph{junction flow ratio}, which gives a measure of how busy the intersection really is, can be calculated as the sum of the critical flow ratios over all phases of the signal cycle:
\eq[.]{e:juncflowratio}{
\flowratio_j = \sum_{p \in \phaset_j} \flowratio_p
} 


\subsubsection*{Lost Time}
Driver reactions are not instantaneous, and vehicles take a finite amount of time to
accelerate and clear the junction. This implies that a non-negligible share of the signal cycle goes wasted, since demand is not served efficiently during the phase transitions:
\begin{itemize}
\item at every phase start, a few seconds pass before vehicles can flow at full capacity, causing a \emph{start-up time loss};
\item at every phase end, sufficient time must be allowed for vehicles to clear the junction before others may safely carry out a conflicting manoeuvre, which represents a \emph{clearance loss}.
\end{itemize}

The start-up loss may be reduced by helping drivers to react more promptly, e.g. using a pre-green amber light or red count-down timers, which also seem to alleviate the stress of being stuck in a queue \todo{ref}.
The clearance loss may only be mitigated by an accurate choice of signal phase sequence for given traffic conditions or, wherever possible, by appropriate modification of the junction layout, e.g. implementation of protected turn bays.

The total lost time $\tlost{j}$ then depends on phase design and sequence, which in turn should
be tailored to the geometry of junction $j$ in relation to the expected traffic conditions.
Each phase contributes its own time losses $\tlost{j,p}$ to the total lost time, which may be
quantified by the following relation between the effective phase green and the phase duration:
\eq[.]{e:phasetimelost}{
\tlost{p} = t_p - \gren{p}
}
The total time loss and the total effective green thus account for the whole signal cycle period:
\eq[.]{e:totloss}{\cycle{j} = \tlost{j} + \sum_{p \in \phaset_j} \gren{p}} 


\subsection{Formulation of the Signal Setting Problem}
Conflicting sets of manoeuvres compete for the right of way at road intersections, and the
main purpose of signalization is to distribute the junction capacity amongst them.

It follows naturally that the allocation of green time to signal phases is the single most important step in signal setting: the cycle must be allotted according to the relative distribution of demand,
lest the junction capacity go wasted and unnecessary queues form on critical approaches.

As far as fixed timing is concerned, optimal allocation of green time is a straightforward
process, yet it can be undertaken according to a number of different principles: early studies
aimed to develop analytical equations, while modern simulation based methods rely on
heuristics to shape the signal setting around a cost function that formalises the chosen signal
setting policy. The next sections provide a general formulation of the problem and a few examples of objective implementation through different setting policies.

\subsubsection*{Lagrangian Formulation}
\newcommand{\grenvec}{\vec{\gren{}}_{\phaset_j}}
The Signal Setting of junction $j$ can be formulated as an optimisation problem, i.e. to find
effective green durations for each phase and cycle time that minimise an objective function while complying with a set of constraints.

A popular choice of cost function may be the average delay at the intersection, given by the weighted average vehicle delay $\avgdelay{a}$ on all lane groups.\\Delay on each lane depends according to equation \req{e:avgdelay} on effective green shares, cycle length, and the relevant flows $\flow_a$ as illustrated in section \ref{s:performance}.

For average delay optimisation of a junction $j$, consider a well-designed phase sequence $\phaset_j$ ensuring minimal conflicts and time losses. The signal program is then fully characterised by a vector of effective phase green shares $\grenvec \in \mathds{R}^{|\phaset_j|}$ together with the cycle time $\cycle{j}$.

The problem takes the following form:
\eq{e:lagrangian}{
\begin{array}{lrl}
\begin{array}{c}
\textbf{min} \\
\grenvec , \: \cycle{j}
\end{array}
& \avgdelay{j} = & \displaystyle \sum_{a \in \bstar{j}} \avgdelay{a} \flow_a
\\ \\
\text{subject to} & \cycle{j} - \tlost{j} = & \displaystyle \sum_{p \in \phaset_j} \gren{p}
\\ \\
 & \gren{p} \geq & \flowratio_p \: \cycle{j} \quad \forall p \in \phaset_j \\
\end{array}
}


\subsubsection*{•}


\section{Signal Coordination}
\fig{htbp}{PIX/greenwave.png}{f:earlysync}{Early signal synchronisation along a San Francisco arterial road, circa 1929. Bands A through T represent vehicle platoons \protect\footnotemark .}{width=\textwidth}
\footnotetext{By City of San Francisco - Public domain (via Eric Fischer), CC BY-SA 3.0,\\ \url{https://commons.wikimedia.org/w/index.php?curid=34715929}}

Traffic light coordination between adjacent junctions is an essential aspect of an
optimal signalisation plan, with disposition of \emph{green waves} as its most notable and popular
feature. Traffic in fact mostly travels along a limited number of main corridors, commonly referred
to as \emph{arteries} carrying \emph{arterial traffic}.

It has long been accepted as a reasonable compromise to minimise user discomfort along those, rather than taking on the much more intricate problem of reducing the total network delay.
Although, undeniably, being able to drive through a streak of green signals already goes a long way towards
improving the quality of a trip from the user point of view, signal coordination chiefly
serves the purpose of ensuring an efficient use of the available infrastructure.

It is in fact of the utmost importance to avoid unnecessary signal-induced delays and stops which could rapidly bring traffic to a grinding halt, even under rather mild conditions which the network could otherwise cope with.

The search for a coordination solution that maximises usability of urban arteries under specific traffic conditions is still mostly carried out offline — as it was for the first attempts at smart arterial signalization, such as the pen-and-paper method portrayed in Figure \ref{f:earlysync}.
To this end, a wide variety of methods have been the object of intensive research since the early 1980s, ranging from simple analytical approaches to heuristics.

Analytical methods have brought about a number of popular applications which are still in use despite the fact that they mainly apply to low congestion scenarios; more complex methods, which account for demand flows and their propagation along the arterial, can deal with heavy congestion related phenomena, but invariably require a more detailed network model and rely on computationally demanding simulations rather than a closed-form problem formulation. An overview of the most prominent approaches to the signal coordination problem is given in the following sections.

\subsection{The Traffic Corridor} \label{s:corridor}
The fulcrum of signal coordination is the \emph{traffic corridor} (i.e. an arterial road, as defined in the previous section) selected for its strategic relevance. Since the flow on the corridor is supposedly much higher than on its side roads, it is deemed acceptable to concentrate optimisation efforts on the arterial traffic conditions, as improvements will benefit the largest number of road users.

A traffic corridor $\corridor$ may be defined as an \emph{ordered} set of $n$ \emph{connected} arcs:
\eq[.]{e:corridordef}{
\arcset \supset \corridor = \left\lbrace a_1, a_2, \dotsc , a_n \right\rbrace \qquad \mathrm{with} 
\quad \left\lbrace
\begin{array}{l}
a_{i-1} \in \bstar{a_i} \quad \forall i > 1 \\
a_{i+1} \in \fstar{a_i} \quad \forall i < n \\
\end{array}
\right.
}

Although all nodes along the corridor are, strictly speaking, junctions, it makes sense in this context to define the ordered subset $\junset_\corridor$ of the $m$ \emph{signalised} junctions that actually regulate the flow on the corridor.
This may be formalised as
\eq{e:corridorjuncs}{
\bigcup_{a \in C} \left\lbrace \tail{a}, \head{a} \right\rbrace \supset
\junset_\corridor = \left\lbrace j_1, j_2, \dotsc , j_m \right\rbrace
\quad \text{such that} \quad \forall j \in \junset_\corridor \quad
\exists y \in \phaset_j | \{\bstar{y}, \fstar{y}\} \subset \corridor
}
where it is simply stated that a corridor node is considered a relevant \emph{signalised junction} if features one \emph{signalised} manoeuvre $y \in \phaset_j$ whose origin and destination lanes $\{\bstar{y}, \fstar{y}\}$ both lie on the corridor (with the exception of the first node of the corridor, which may be included in $\junset_\corridor$ as long as it regulates at least one turn onto the corridor, and the last one if the corridor outflow may be affected by its signal).

Coordination of junctions $\junset_\corridor$ is handled by offsetting their local timing instructions (as described at the end of section \ref{s:anatomy}), i.e. anticipating or delaying all phase changes rigidly without altering the necessary green shares determined on the basis of average demand flows.
The global offset values (with respect to an arbitrary global time reference) of the junctions of corridor $\corridor$ may be represented by a vector $\offsvec{\corridor}$.

Furthermore, it is assumed that all junctions of the corridor share the same cycle time, so that in the context of signal coordination the symbol $\cycle{\corridor}$ refers to all junctions, and may be even used without the subscript $\corridor$.


\subsection{Bandwidth Maximisation} \label{s:bandmax}
In relation to arterial traffic, the concept of \emph{progression bandwidth} emerges as a measure of the quality of a green wave setup along a \emph{corridor} and can be defined as the duration of the time window through which a vehicle may enter the artery and travel its entire length without encountering red lights nor standing queues.

By reducing delays and number of stops along the most critical paths, bandwidth maximisation is a relatively straightforward but effective way to help the system meet user expectations about traffic fluidity, mitigating the stress associated with driving in a congested urban environment. Moreover, this type of signal coordination has proven highly beneficial in reducing the chance of rear end collisions and red signal violations \todo{ref (Li, Tarko 2010)} as well as pollution levels associated with the hiccupping stop-and-go driving often experienced under poorly coordinated signalisation.

Bandwidth maximisation has been formulated as a Linear Optimisation problem since \todo{ref Little et al. (1981)} which led to development of the MAXBAND/MULTIBAND series of software solutions. These considered the offsets between junctions as the only decision variables, but provided a computationally viable method for one-way and two-way bandwidth maximisation relying solely on the target travel times between junctions and predetermined signal cycle length and green times. \todo{A more efficient solution method was introduced by ... ref Papola and Fusco (1988a) ... expand}.

However, relevant discrepancies — dubbed \emph{bandwidth degradation} — were observed between the expected outcome and the real-world performance of the signal plans generated by these early methods. It is now universally accepted that, as \todo{ref Tsay and Lin (1988)} amongst many others pointed out, the underlying models were oversimplified and no account was taken of side flows and platoon dispersion. 

Proposed extensions of the original method aimed to factor in queue and side flow clearance times, to produce a more realistic bandwidth model for phase offset determination. The analytical relationship between maximal bandwidth and minimum delay problems was finally formulated in \todo{Papola and Fusco (2000) ... analytical model, delays as function of the maximal bandwidth and other variables}.

At present, offline arterial progression optimisation techniques invariably rely on some formulation of the \emph{bandwidth maximisation problem} (as in the cases illustrated in the next section), which is to say that their common objective is to maximise a \emph{theoretical} traffic throughput, often without much consideration for network performance \todo{(and online too! the point is: nobody looks beyond the throughput but we are trying to)}.

\todo{Make it clear that the slack band concept is unorthodox but original, and implemented in the context of this work.}


\subsubsection*{Mixed Integer Linear Programming Approach}
\todo{Illustrate De Nunzio 2015 Variable Speed Limit Bandwidth LP optimiser}

\subsubsection*{The Slack Band Approach}
\todo{Illustrate the slack bandwidth generalisation.}
The idea of \emph{slack} bandwidth is an answer to the very strict definition of bandwidth given at the beginning of section \ref{s:bandmax}, according to which only the band running through all junctions counts for something, implying that:
\begin{itemize}
\item if one passing phase is particularly short, coordination between longer green phases may be disregarded \todo{(discuss fringes in MILP approach)};
\item in bi-directional optimisation, maximisation of the return band may prioritise a very narrow band that \emph{just} makes it through all junctions (possibly degrading the main band significantly) over a very wide band divided in two or more chunks.
\end{itemize} 
\todo{To avoid this we consider the total band between any pair of junctions: illustrate algorithm.}

\section{Advanced Offline Signal Planning}
\todo{Describe heuristic approaches, and complex offline performance optimisation. Anticipate how the present work will build and improve upon those. TO REVIEW.}

The simple signal setting problems presented so far are quasi-convex, but more realistic traffic models that include and quantify global performance indicators such as total delay introduce an inherent non convexity, better addressed with the aid of heuristic methods.
With the increase in computing power availability, metaheuristics have seen a substantial rise in popularity as means to overcome the inherent limitations of analytical formulations: heuristic approaches to this class of problems involve the generation of a large — yet manageable, compared to the dimensions of the search space — number of candidate timing solutions, the effects of which are then simulated to evaluate their fitness. At each iteration, a variety of methods ranging from Genetic Algorithms to Simulated Annealing and Particle Swarm Optimisation can then be used to modify and combine the most successful solutions into a new set of candidates.
Such methods are particularly suited for solving obscure problems as they require no attempt to establish an explicit correlation between the control variables and the desired outcome. Rather, they rely on the assumption that if any relevant phenomena can be modelled with sufficient accuracy and a performance index can describe the degree of achievement of the optimization objectives, then the system can be made to naturally evolve towards an optimal solution.

\fig{htbp}{pix/heur.png}{f:heuristic}{Conceptual information flow in a heuristic approach to signal optimisation}{width=0.65\textwidth}

It is therefore obvious that the model used to assess the fitness of candidate solutions should represent a sensible trade-off between speed and completeness: the real-world performance will inevitably be disappointing if the optimisation could not account for relevant traffic phenomena that were simplified out of the solution assessment, while on the other hand the need to evaluate huge numbers of candidate solutions calls for a lean and fast method to predict the outcome of a given timing plan. Heuristics that depend heavily on the choice of initial conditions often use maximum bandwidth solutions as starting point in the search for minimum total delay, to shave off convergence time and increase the quality and applicability of solutions.

This approach has been taken most notably by the Transport Research Laboratory, the UK based institution that since Robertson (1969) has been developing the TRAffic Network StudY Tool, which was born as a software tool to minimise stops along arterial roads while accounting for reasonably realistic vehicle behaviour, and was gradually extended to model ever more complex phenomena. Today, TRANSYT can handle pedestrian flows, optimise green shares as well as junction offsets and include actuated signals, all the while monitoring a custom set of network-wide performance indicators that can implement whatever policy the traffic administration desires. The optimisation relies on the availability of a complete transportation network model, possibly including detailed junction geometry. A range of search algorithms can be used to explore complex timing solutions, which are then evaluated using either micro– or macrosimulation models.
Earliest version of Transyt implemented a simple hill climbing algorithm that explored the non convex performance function by executing a predetermined set of short and long steps to vary each control variable in both directions alternatively. At each step, the changed value of the control variable is kept if it improved the performance index. Park et al. (1999) introduced a traffic signal optimization program for oversaturated intersections consisting of two modules: a genetic algorithm optimizer and mesoscopic simulator. Colombaroni et al. (2009) devised a solution procedure that first applies a genetic algorithm and then a hill climbing algorithm for local adjustments. The fitness function is evaluated by means of a traffic model that computes platoon progression along the links, their combination and possible queuing at nodes through analytical delay formulations. Colombaroni et al. (2010) extended the model to design optimal signal settings of a synchronised artery with predetermined rules for dynamic bus priority. A visual example is given in \todo{Figure 7} of the genetic algorithm representation of signal settings as a chromosome population.

Metaheuristics often see applications in traffic signal engineering that reach beyond ordinary signal planning, and have more than once played an important role in research by aiding the formalisation of less intuitive correlations between signal settings and traffic behaviour. Gentile and Tiddi (2009) use a Genetic Algorithm to venture out into the yet uncharted territory of arterial synchronisation under heavy congestion and queue spillback. To predict the outcome of candidate signal plans, the heuristic method relies on the General Link Transmission Model (Gentile 2015), which implements the Kinematic Wave Theory to allow accurate simulation of traffic dynamics and model physical blockage of links, while requiring sufficiently short computation times to deal with the very large number of solutions to be evaluated.
The optimisation reveals a crucial difference between subcritical and supercritical flow conditions: while in the former case the optimal green wave is led as usual by the flow velocity, the same approach proves completely ineffective under supercritical conditions, which oppositely demand that the backwards propagating jam wave speed should set the pace of upstream signals, to ensure that the residual capacity of saturated links is fully exploited.

It must be noted that the level of detail taken into account when using metaheuristics comes at a heavy cost in terms of computation speed, which restricts \todo{(has so far restricted! reformulate around present work)} the functionality of this type of software to that of advanced yet offline development tools. As long as ordinarily accessible computing power remains insufficient for true real time functionality, advanced optimization suites are staying on top of the game by attempting to streamline the interactions between the development environment and the street-level equipment, e.g. providing offline optimisation based on real time readings and quick and simple deployment of new plans.

Such efforts are driving this type of software towards a sort of mid-term, day-to-day real time adaptivity which, possibly in combination with a true real-time program selection logic may well prove to be an effective approach to \todo{slowly and steadily improve the signalization of large urban networks (meh, cut)}.