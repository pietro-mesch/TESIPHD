\section{Signal Coordination}
\fig{htbp}{PIX/greenwave.png}{f:earlysync}{Early signal synchronisation along a San Francisco arterial road, circa 1929. Bands A through T represent vehicle platoons \protect\footnotemark .}{width=\textwidth}
\footnotetext{By City of San Francisco - Public domain (via Eric Fischer), CC BY-SA 3.0,\\ \url{https://commons.wikimedia.org/w/index.php?curid=34715929}}

Traffic light coordination between adjacent junctions is an essential aspect of an
optimal signalisation plan, with disposition of \emph{green waves} as its most notable and popular
feature. Traffic in fact mostly travels along a limited number of main corridors, commonly referred
to as \emph{arteries} carrying \emph{arterial traffic}.

It has long been accepted as a reasonable compromise to minimise user discomfort along those, rather than taking on the much more intricate problem of reducing the total network delay.
Although, undeniably, being able to drive through a streak of green signals already goes a long way towards
improving the quality of a trip from the user point of view, signal coordination chiefly
serves the purpose of ensuring an efficient use of the available infrastructure.

It is in fact of the utmost importance to avoid unnecessary signal-induced delays and stops which could rapidly bring traffic to a grinding halt, even under rather mild conditions which the network could otherwise cope with.

The search for a coordination solution that maximises usability of urban arteries under specific traffic conditions is still mostly carried out offline — as it was for the first attempts at smart arterial signalization, such as the pen-and-paper method portrayed in Figure \ref{f:earlysync}.
To this end, a wide variety of methods have been the object of intensive research since the early 1980s, ranging from simple analytical approaches to heuristics.

Analytical methods have brought about a number of popular applications which are still in use despite the fact that they mainly apply to low congestion scenarios; more complex methods, which account for demand flows and their propagation along the arterial, can deal with heavy congestion related phenomena, but invariably require a more detailed network model and rely on computationally demanding simulations rather than a closed-form problem formulation. An overview of the most prominent approaches to the signal coordination problem is given in the following sections.

\subsection{The Traffic Corridor} \label{s:corridor}
The fulcrum of signal coordination is the \emph{traffic corridor} (i.e. an arterial road, as defined in the previous section) selected for its strategic relevance. Since the flow on the corridor is supposedly much higher than on its side roads, it is deemed acceptable to concentrate optimisation efforts on the arterial traffic conditions, as improvements will benefit the largest number of road users.

A traffic corridor $\corridor$ may be defined as an \emph{ordered} set of $n$ \emph{connected} arcs:
\eq[.]{e:corridordef}{
\arcset \supset \corridor = \left\lbrace a_1, a_2, \dotsc , a_n \right\rbrace \qquad \mathrm{with} 
\quad \left\lbrace
\begin{array}{l}
a_{i-1} \in \bstar{a_i} \quad \forall i > 1 \\
a_{i+1} \in \fstar{a_i} \quad \forall i < n \\
\end{array}
\right.
}

Although all nodes along the corridor are, strictly speaking, junctions, it makes sense in this context to define the ordered subset $\junset_\corridor$ of the $m$ \emph{signalised} junctions that actually regulate the flow on the corridor.
This may be formalised as
\eq{e:corridorjuncs}{
\bigcup_{a \in C} \left\lbrace \tail{a}, \head{a} \right\rbrace \supset
\junset_\corridor = \left\lbrace j_1, j_2, \dotsc , j_m \right\rbrace
\quad \text{such that} \quad \forall j \in \junset_\corridor \quad
\exists y \in \phaset_j | \{\bstar{y}, \fstar{y}\} \subset \corridor
}
where it is simply stated that a corridor node is considered a relevant \emph{signalised junction} if features one \emph{signalised} manoeuvre $y \in \phaset_j$ whose origin and destination lanes $\{\bstar{y}, \fstar{y}\}$ both lie on the corridor (with the exception of the first node of the corridor, which may be included in $\junset_\corridor$ as long as it regulates at least one turn onto the corridor, and the last one if the corridor outflow may be affected by its signal).

Coordination of junctions $\junset_\corridor$ is handled by offsetting their local timing instructions (as described at the end of section \ref{s:anatomy}), i.e. anticipating or delaying all phase changes rigidly without altering the necessary green shares determined on the basis of average demand flows.
The global offset values (with respect to an arbitrary global time reference) of the junctions of corridor $\corridor$ may be represented by a vector $\offsvec{\corridor}$.

Furthermore, it is assumed that all junctions of the corridor share the same cycle time, so that in the context of signal coordination the symbol $\cycle{\corridor}$ refers to all junctions, and may be even used without the subscript $\corridor$.


\subsection{Bandwidth Maximisation} \label{s:bandmax}
In relation to arterial traffic, the concept of \emph{progression bandwidth} emerges as a measure of the quality of a green wave setup along a \emph{corridor} and can be defined as the duration of the time window through which a vehicle may enter the artery and travel its entire length without encountering red lights nor standing queues.

By reducing delays and number of stops along the most critical paths, bandwidth maximisation is a relatively straightforward but effective way to help the system meet user expectations about traffic fluidity, mitigating the stress associated with driving in a congested urban environment. Moreover, this type of signal coordination has proven highly beneficial in reducing the chance of rear end collisions and red signal violations \todo{ref (Li, Tarko 2010)} as well as pollution levels associated with the hiccupping stop-and-go driving often experienced under poorly coordinated signalisation.

Bandwidth maximisation has been formulated as a Linear Optimisation problem since \todo{ref Little et al. (1981)} which led to development of the MAXBAND/MULTIBAND series of software solutions. These considered the offsets between junctions as the only decision variables, but provided a computationally viable method for one-way and two-way bandwidth maximisation relying solely on the target travel times between junctions and predetermined signal cycle length and green times. \todo{A more efficient solution method was introduced by ... ref Papola and Fusco (1988a) ... expand}.

However, relevant discrepancies — dubbed \emph{bandwidth degradation} — were observed between the expected outcome and the real-world performance of the signal plans generated by these early methods. It is now universally accepted that, as \todo{ref Tsay and Lin (1988)} amongst many others pointed out, the underlying models were oversimplified and no account was taken of side flows and platoon dispersion. 

Proposed extensions of the original method aimed to factor in queue and side flow clearance times, to produce a more realistic bandwidth model for phase offset determination. The analytical relationship between maximal bandwidth and minimum delay problems was finally formulated in \todo{Papola and Fusco (2000) ... analytical model, delays as function of the maximal bandwidth and other variables}.

At present, offline arterial progression optimisation techniques invariably rely on some formulation of the \emph{bandwidth maximisation problem} (as in the cases illustrated in the next section), which is to say that their common objective is to maximise a \emph{theoretical} traffic throughput, often without much consideration for network performance \todo{(and online too! the point is: nobody looks beyond the throughput but we are trying to)}.

\todo{Make it clear that the slack band concept is unorthodox but original, and implemented in the context of this work.}


\subsubsection*{Mixed Integer Linear Programming Approach}
\todo{Illustrate De Nunzio 2015 Variable Speed Limit Bandwidth LP optimiser}

\subsubsection*{The Slack Band Approach}
\todo{Illustrate the slack bandwidth generalisation.}
The idea of \emph{slack} bandwidth is an answer to the very strict definition of bandwidth given at the beginning of section \ref{s:bandmax}, according to which only the band running through all junctions counts for something, implying that:
\begin{itemize}
\item if one passing phase is particularly short, coordination between longer green phases may be disregarded \todo{(discuss fringes in MILP approach)};
\item in bi-directional optimisation, maximisation of the return band may prioritise a very narrow band that \emph{just} makes it through all junctions (possibly degrading the main band significantly) over a very wide band divided in two or more chunks.
\end{itemize} 
\todo{To avoid this we consider the total band between any pair of junctions: illustrate algorithm.}