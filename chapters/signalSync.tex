\section{Signal Coordination}
\fig{htbp}{PIX/greenwave.png}{f:earlysync}{Early signal synchronisation along a San Francisco arterial road, circa 1929. Bands A through T represent vehicle platoons \protect\footnotemark .}{width=\textwidth}
\footnotetext{By City of San Francisco - Public domain (via Eric Fischer), CC BY-SA 3.0,\\ \url{https://commons.wikimedia.org/w/index.php?curid=34715929}}

Traffic light coordination between adjacent junctions is an essential aspect of an
optimal signalisation plan, with disposition of \emph{green waves} as its most notable and popular
feature. Traffic in fact mostly travels along a limited number of main corridors, commonly referred
to as \emph{arteries} carrying \emph{arterial traffic}.

It has long been accepted as a reasonable compromise to minimise user discomfort along those, rather than taking on the much more intricate problem of reducing the total network delay.
Although, undeniably, being able to drive through a streak of green signals already goes a long way towards
improving the quality of a trip from the user point of view, signal coordination chiefly
serves the purpose of ensuring an efficient use of the available infrastructure.

It is in fact of the utmost importance to avoid unnecessary signal-induced delays and stops which could rapidly bring traffic to a grinding halt, even under rather mild conditions which the network could otherwise cope with.

The search for a coordination solution that maximises usability of urban arteries under specific traffic conditions is still mostly carried out offline — as it was for the first attempts at smart arterial signalization, such as the pen-and-paper method portrayed in Figure \ref{f:earlysync}.
To this end, a wide variety of methods have been the object of intensive research since the early 1980s, ranging from simple analytical approaches to heuristics.

Analytical methods have brought about a number of popular applications which are still in use despite the fact that they mainly apply to low congestion scenarios; more complex methods, which account for demand flows and their propagation along the arterial, can deal with heavy congestion related phenomena, but invariably require a more detailed network model and rely on computationally demanding simulations rather than a closed-form problem formulation. An overview of the most prominent approaches to the signal coordination problem is given in the following sections.

\subsection{The Traffic Corridor} \label{s:corridor}
The fulcrum of signal coordination is the \emph{traffic corridor} (i.e. an arterial road, as defined in the previous section) selected for its strategic relevance. Since the flow on the corridor is supposedly much higher than on its side roads, it is deemed acceptable to concentrate optimisation efforts on the arterial traffic conditions, as improvements will benefit the largest number of road users.

A traffic corridor $\corridor$ may be defined as an \emph{ordered} set of $n$ \emph{connected} arcs:
\eq[.]{e:corridordef}{
\arcset \supset \corridor = \left\lbrace a_1, a_2, \dotsc , a_n \right\rbrace \qquad \mathrm{with} 
\quad \left\lbrace
\begin{array}{l}
a_{i-1} \in \bstar{a_i} \quad \forall i > 1 \\
a_{i+1} \in \fstar{a_i} \quad \forall i < n \\
\end{array}
\right.
}

Although all nodes along the corridor are, strictly speaking, junctions, it makes sense in this context to define the ordered subset $\junset_\corridor$ of the $m$ \emph{signalised} junctions that actually regulate the flow on the corridor.
This may be formalised as
\eq{e:corridorjuncs}{
\bigcup_{a \in C} \left\lbrace \tail{a}, \head{a} \right\rbrace \supset
\junset_\corridor = \left\lbrace j_1, j_2, \dotsc , j_m \right\rbrace
\quad \text{such that} \quad \forall j \in \junset_\corridor \quad
\exists y \in \phaset_j | \{\bstar{y}, \fstar{y}\} \subset \corridor
}
where it is simply stated that a corridor node is considered a relevant \emph{signalised junction} if features one \emph{signalised} manoeuvre $y \in \phaset_j$ whose origin and destination lanes $\{\bstar{y}, \fstar{y}\}$ both lie on the corridor (with the exception of the first node of the corridor, which may be included in $\junset_\corridor$ as long as it regulates at least one turn onto the corridor, and the last one if the corridor outflow may be affected by its signal).

Coordination of junctions $\junset_\corridor$ is handled by offsetting their local timing instructions (as described at the end of section \ref{s:anatomy}), i.e. anticipating or delaying all phase changes rigidly without altering the necessary green shares determined on the basis of average demand flows.
The global offset values (with respect to an arbitrary global time reference) of the junctions of corridor $\corridor$ may be represented by a vector $\offsvec{\corridor}$.

Furthermore, it is assumed that all junctions of the corridor share the same cycle time, so that in the context of signal coordination the symbol $\cycle{\corridor}$ refers to all junctions, and may be even used without the subscript $\corridor$.


\subsection{Bandwidth Maximisation} \label{s:bandmax}
In relation to arterial traffic, the concept of \emph{progression bandwidth} emerges as a measure of the quality of a green wave setup along a \emph{corridor} and can be defined as the duration of the time window through which a vehicle may enter the artery and travel its entire length without encountering red lights nor standing queues.

By reducing delays and number of stops along the most critical paths, bandwidth maximisation is a relatively straightforward but effective way to help the system meet user expectations about traffic fluidity, mitigating the stress associated with driving in a congested urban environment. Moreover, this type of signal coordination has proven highly beneficial in reducing the chance of rear end collisions and red signal violations \citep{li2010safety} as well as pollution levels associated with the hiccupping stop-and-go driving often experienced under poorly coordinated signalisation.

Bandwidth maximisation has been formulated as a Linear Optimisation problem since \citep{little1981maxband} which led to development of the MAXBAND/MULTIBAND series of software solutions. These considered the offsets between junctions as the only decision variables, but provided a computationally viable method for one-way and two-way bandwidth maximisation relying solely on the target travel times between junctions and predetermined signal cycle length and green times.

However, relevant discrepancies — dubbed \emph{bandwidth degradation} — were observed between the expected outcome and the real-world performance of the signal plans generated by these early methods: it is now universally accepted that, as \citep{tsay1988new} amongst many others pointed out, the underlying models were oversimplified and no account was taken of side flows and platoon dispersion. 

Proposed extensions of the original method aimed to factor in queue and side flow clearance times, to produce a more realistic bandwidth model for phase offset determination. The analytical relationship between maximal bandwidth and minimum delay problems was finally formalised in \citep{papola2000new}, where travel times and delays are expressed as a function of the maximal bandwidth and other variables accounting for the entity of side flows, interstage sequences etc.

At present, offline arterial progression optimisation techniques invariably rely on some formulation of the \emph{bandwidth maximisation problem} (as in the cases illustrated in the next section), which is to say that their common objective is to maximise a \emph{theoretical} traffic throughput, often without much consideration for network performance. This is also true for \emph{online} optimisation tools that evaluate signal plan updates with a similar goal, ignoring the fact that traffic propagation is a rather complex phenomenon which has the utmost relevance upon bandwith degradation: as explained in detail in Chapter \ref{c:optimiser}, one of the foremost aims of this work is to renounce the geometric formalisation of bandwidth as a measure of progression in favour of a simulation based approach, to better reproduce the relation between coordination and queue dynamics, and possibly look past the long standing preconception that to maximise progression identifies with optimal operation conditions for any road under any circumstance.

The next sections illustrate two numerical approaches to the complex problem of two-way bandwidth maximisation: the first is an elegant implementation of the classic paradigm of progression optimisation, fully featured in closed form and solved as a linear program with the addition of variable speed limits; the second is a much simpler yet effective geometrical method developed in the context of this work.

\subsubsection*{Mixed Integer Linear Programming Approach}


One of the most complete and effective implementations of the MILP approach to two-way arterial coordination is presented in \citep{de2015arterial}. The bandwidth maximisation was extended to include \emph{Variable Speed Limits} (VSL) as control variables, allowing for a wider range of high bandwidth solutions. The method is outlined here as an example of the degree of complexity that can be managed by mathematical optimisation.

The concept of VSL has been applied to motorway traffic for quite some time, to enhance traffic fluidity in response to congestion, accidents or adverse weather, but its application to urban traffic presents new challenges, not least the need for effective means of introducing it and getting it across to the drivers: in pilot projects this is quite effectively achieved by variable led panels, mimicking an ordinary speed limit sign, showing the target synchronisation speed. Were such measures to gain popularity, the already promising degree of driver compliance can only be expected to improve.

The simple MILP approach to two-way bandwidth maximisation can be summarised by considering a corridor $\corridor$ (see section \ref{s:corridor}) running through an ordered set of intersections $\junset_{\corridor}$ along its \emph{main} driving direction, while the opposite,
possibly lower priority direction traverses the same nodes in reverse order.

If positive travel speeds $v_j$ and $\bar{v}_j$ are defined between $j$ and $j+1$,
in the main and return direction respectively, and a generic spatial coordinate $x$ is considered, increasing along the corridor with the index $j$, travel times for perfect green waves should then be:

\eq[.]{e:milp1}{
\left\lbrace
\begin{array}{l}
t_j = \frac{x_{j+1} - x_j}{v_j} > 0
\vspace{3pt}\\
\bar{t}_j = \frac{x_{j} - x_{j+1}}{\bar{v}_j} < 0
\end{array}
\right.
\qquad
\forall j \in \left[ 1 \, , |\junset_\corridor|-1 \right]
}

Assuming a common cycle time $\cycle{\corridor}$, consider at each node and for both directions:
\begin{description}
\item[effective green duration] of the arterial \emph{through} movement phases $\gren{j}$ and $\bar{\gren{j}}$;
\item[absolute offset] as the time between the midpoint of a green phase and the closest multiple of the cycle time $\toffs{j}, \bar{\toffs{j}}$.
\end{description}

A nonstandard modulo operation $\tmod{\bullet}$ can be defined for brevity, to refer \emph{any} time to the corridor cycle, such that
\eq{e:milp2}{\tmod{t^*}
\in \left] 
-\frac{\cycle{}}{2} , \frac{\cycle{}}{2} \right]
}
returns the distance from $t^*$ to the nearest multiple of $\cycle{}$.

The modulo is used to define the \emph{internal offset} given by
\eq{e:milp3}{
\intoffs{j} = \tmod{\bar{\toffs{j}} - \toffs{j}}
}
and the \emph{relative offset} $\reloffs{j}$. The latter represents the time coordinate of the mid-green instant of the relevant phase with respect to a \emph{moving} frame of reference travelling along the \emph{main} driving direction, starting in $x_1$ at the zero instant and moving with the specified speeds $v_j$ between nodes.

Hence, the relative offset at each node after the first can be computed easily from
the offsets at upstream nodes:
\eq[.]{e:milp4}{
\begin{array}{rl}
\toffs{j} - \reloffs{j} = \toffs{j-1} - \reloffs{j-1} + t_{j-1}
\quad \Rightarrow & \quad
\reloffs{j} = \tmod{\reloffs{j-1} + \toffs{j} - \toffs{j-1} - t_{j-1}} \vspace{5pt} \\ 
\Rightarrow &
\begin{cases}
\reloffs{j} = \tmod{\reloffs{1} - \toffs{1} + \toffs{j} - \displaystyle\sum_{i=1}^{j-1} t_i} 
\vspace{3pt}\\
\toffs{j} = \tmod{\toffs{1} - \reloffs{1} + \reloffs{j} + \displaystyle\sum_{i=1}^{j-1} t_i}
\end{cases}
\end{array}
}

In order to express the bandwith in both directions in terms of the relative offsets, it is
also beneficial to map all $\intoffs{j}$ to the time reference of the first junction using
\eq[,]{e:milp5}{\intoffz{j} =
\tmod{\intoffs{j} + \sum_{i=1}^{j-1}\left(t_i - \bar{t}_i \right) }}

and considering that the $\intoffs{j}$ are described by the signal program at each intersection, which leads to the vector equation linking the offsets in the two directions
\eq[.]{e:milp6}{
\mathbf{\bar{t}^\Delta = t^\Delta - t^\delta} \qquad \text{with} \;
\begin{cases}
\mathbf{t^\Delta} = 
\left( \reloffs{1}, \reloffs{2}, \dots , \reloffs{| \junset_\corridor |} \right) \\
\mathbf{t^\delta} =
\left( \intoffz{1}, \intoffz{1}-\intoffz{2}, \dots ,
\intoffz{1} - \intoffz{| \junset_\corridor |} \right)
\end{cases}
}

\fig{htb}{PIX/bandwidth.png}{f:milp}{
Bandwidth Problem Formulation: the signal coordination parameters are portrayed on a
distance-time (D-T) graph. Temporal references are given by integer multiples of the cycle time and by the synchronisation frame of reference, moving along the diagonal trajectories at speeds $v_j$. The green phase in the main direction is drawn on the left of each junction’s temporal line, that of the inverse direction is to its right. Notice the offsets measured between the phase midpoints and the time of arrival of the moving FoR.}{width=\textwidth}

Finally, it is possible to express bandwidth as a function of travel
times and offsets. According to the definition given at the start of this section and considering
Figure \ref{f:milp}, it is the intersection of all green windows as seen in the moving frame of reference:
\eq[.]{e:milp7}{
\bigcap_{j \in \junset_\corridor} 
\left[
\reloffs{j} - \frac{\gren{j}}{2} \; , \; \reloffs{j} + \frac{\gren{j}}{2}
\right]
}

The bandwidth value in the \emph{main} direction is then calculated from the decision variables as
\eq{e:hardband}{
\hardband{\corridor} = \hardband{} \left(\mathbf{\reloffs{}} \right) = 
\min \left\lbrace \left( \reloffs{i} - \reloffs{j} + \gren{ij} \right) \quad \forall i,j \in \junset_\corridor \right\rbrace
\qquad \text{with} \;
\gren{ij} = \frac{\gren{i} + \gren{j}}{2}
}
which is the \emph{smallest possible} overlap between \emph{any two} green phases in the moving FoR; the equivalent in the other direction is found using the relevant green times $\bar{\gren{}}$ and the relation given by \req{e:milp6}.

The sum of the bandwidths in the two directions can then be the objective of the linear
optimiser — bounded by appropriate constraints such as maximum speed values — in conjunction with any function of the decision variables used to favour a certain type of solution: for example, the optimisation presented in \citep{de2015arterial} is driven by an extended utility function aiming to favour low travel times and minimise the speed indication variance across segments so as to ease drivers into complying with apparently arbitrary limits.

Real world statistics are beginning to back up the simulation results that originally validated these studies, proving the following interesting points about modern bandwidth maximisation techniques:
\begin{itemize}
\item the best combinations of optimal offsets and VSL drastically reduce the number of stops and energy consumption;
\item lower and smoother speed limits reduce energy consumption at \emph{no disadvantage }to the total arterial travel time;
\item VSL brings about larger bandwidth and faster solution of the LP.
\end{itemize}

It must be noted however that despite the practically negligible computation times associated with the LP methods just mentioned, these remain conceptually unfit for \emph{real time} signal optimisation since they take in no account the flow and speed of the actual traffic, nor they apply outside the safe boundaries of \emph{capacity} conditions (whereby green time is always assumed sufficient to deal with demand).



\subsubsection{The Slack Band Approach} \label{s:slackband}
\todo{Illustrate the slack bandwidth generalisation.}
The idea of \emph{slack} bandwidth is an answer to the very strict definition of bandwidth given at the beginning of section \ref{s:bandmax}, according to which only the band running through all junctions counts for something, implying that:
\begin{itemize}
\item if one passing phase is particularly short, coordination between longer green phases may be disregarded: because of \req{e:hardband} the bandwidth is throttled to be at most as wide as the shortest phase;
\item in bi-directional optimisation, maximisation of the return band may prioritise a very narrow band that \emph{just} makes it through all junctions (possibly degrading the main band significantly) over a very wide band divided in two or more chunks.
\end{itemize}

To avoid such inconveniences, which are intrinsic in the definition of what will henceforth be referred to as the \emph{canonical} bandwidth $\hardband{\corridor}$, the slack bandwidth paradigm attempts to describe the overall \emph{"progressivity"} of the corridor along its whole length, by considering the sum of the individual green bands leading up to and following \emph{any} of the corridor junctions.

Rather than a length of time (the width of $\hardband{\corridor}$ on a T-D graph) the slack bandwidth has dimensions \units{L \cdot T} (an area on a T-D graph) i.e. it is the product of the time during which a vehicle may enter each section of the corridor and the distance it will travel unhindered as a result. If the times are normalised with respect to the cycle time, the slack band becomes a \emph{probability $\times$ distance} product (just as the canonical bandwidth would represent the overall chance of travelling the whole corridor without stopping).

The formalisation of this idea is much simpler than it may sound at first. 
Consider a junction $j$ somewhere along corridor $\corridor$: the \emph{forwards slack progression band} $\fslackband{j}$ is the integral of: the distance $l_j$ that may be travelled without stopping, with respect to the time $t$ at which a vehicle leaves from $j$; the former obviously a function of the latter which may be expressed as $l_j(t)$.

Using a compact definition of the \emph{through} phase at $j$ as the interval during which the corresponding manoeuvres are open
\eq[,]{e:gint}{\gint{j} = \left[ \gini{j} \; , \;\: \gini{j} + \gren{j} \right]}
where $\gini{j}$ is the beginning time of the through phase at $j$ and $\gren{j}$ its duration, as before, the forward band can be expressed as
\eq[.]{e:slackint}{
\fslackband{j} = \int_{\gint{j}} l(t) \; \mathrm{d}t}

The integral in \req{e:slackint} clearly formalises the definition of slack bandwidth illustrated in Figure \ref{f:slackdef}.a but it's not practical to compute, and does not provide an explicit form for $l(t)$.

\fig{htbp}{PIX/undoguy.png}{f:slackdef}{ 
The \emph{forward slack band} definition on T-D diagrams expressed in \textbf{(a)} as the integral \req{e:slackint} and in \textbf{(b)} as the discrete sum \req{e:slacksum}.}{width=0.8\textwidth}

Consider therefore the interval $\gint{ji}$ during which a vehicle that \emph{left} $j$ during $\gint{j}$ may drive through a \emph{subsequent} junction $i \geqslant j$. As the vehicles progress along the corridor, only the ones that reach each junction $i$ during the corresponding through phase $\gint{i}$ can proceed without stopping.

Now, the interval over which vehicles that left $i$ during $\gint{i}$ reach $i+1$ can be expressed as 
\eq[.]{e:ginth}{
\ginth{i} = \left[\inf \gint{i} + t_i \, , \, \sup \gint{i} + t_i \right]}
with a simple forward translation to account for the travel time $t_i$ of the relevant corridor section, whence the passing band can be shown to gradually narrow down by intersection with each subsequent green phase
\eq[.]{e:slacknarrowing}{
\gint{j \, i} = \gint{i} \cap \ginth{j \, i-1}}

Finally, using $|\gint{}|$ to indicate the length of a passing interval, the forward slack band can be calculated recursively from any junction $j$ to the end of the corridor:
\eq[,]{e:slacksum}{
\fslackband{j} = \sum_{i \corridor \geqslant j}
\left| \gint{ji} \right| \cdot \length_{i}
}
considering that at the reference junction the passing interval \emph{is} the green phase $\gint{j\, i=j} = \gint{j}$.

Plainly following the specular process back to the beginning of the corridor, it is possible to calculate the \emph{backwards slack progression band} $\bslackband{j}$, to quantify the chances of a vehicle reaching junction $j$ unhindered by red lights. It is also plain that the subsequent applications of \req{e:slacknarrowing} may well yield an empty intersection before the end or the beginning of the corridor are reached: this is not an issue, as the value of this \emph{something-is-better-than-nothing} approach lies exactly in the ability to consider \emph{any} length that can be travelled without stopping. Some computation time can be saved by checking for this condition and stopping the recursive process \req{e:slacksum} as soon as all vehicles that left $j$ have stopped at $i$ (or, going the other way, as soon as none of the vehicles leaving $i$ will reach $j$ without stopping).

The \emph{total} slack band for a given corridor is the normalised sum of the forwards and backwards bands calculated at each junction:
\eq[.]{e:slackband}{
\slackband{\corridor} = \frac{1}{|\junset_\corridor|} \sum_{j \in \corridor} \bslackband{j} + \fslackband{j}
}
With the formalisation complete, it is worth noting the following aspects about the new metric, also illustrated in Figure \ref{f:slackvhard}:
\begin{itemize}
\item normalisation implies that the method favours letting vehicles onto \emph{longer} arcs, which maximise the product in \req{e:slacksum}, rather than short ones, which are more vulnerable to spillback;
\item if a perfect, continuous green wave can be obtained along the whole corridor, the result is identical to the canonical bandwidth value multiplied by the length of the corridor 
$\slackband{\corridor} = \hardband{\corridor} \cdot \length_\corridor$ (this requires that all green phases also have the same length);
\item in all other cases, the slack band value is \emph{strictly greater} than 
$\hardband{\corridor} \cdot \length_\corridor$ as it factors in all fringes and partial bandwidths that \req{e:milp7} necessarily excludes, which is the main point of this metric.
\end{itemize}

\fig{htbp}{PIX/undoguy.png}{f:slackvhard}{A comparison of the results obtained by optimising the canonical progression bandwidth (left) and the slack band metric (right). Bands crossing the T-D diagram from the bottom-left to the top-right are travelling in the main direction, the others in the secondary direction.\\Darker bands on the slack band diagram are in common between a higher number of junctions, the darkest corresponding to the canonical bandwidth and all others to the fringes and partial green waves that the new metric allows to weigh in.}{width=0.4\textwidth}

With given offsets and signal programs, the computation of this metric is almost instantaneous for any conceivable real-world traffic arterial, allowing to find an optimal solution in seconds using a stochastic search method as described in section \todo{REF}. Its effectiveness confirmed by the results presented in \todo{ReF}, this method was used to find ideal initial conditions for the real-time traffic coordination module presented in this work.