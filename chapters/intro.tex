\chapter*{Introduction}
The fundamental role of \emph{traffic signals} is to equitably and efficiently administer the right of way amongst conflicting streams of road users.

Since the first sporadic appearances around the turn of the 20th century, traffic lights have become a ubiquitous feature in the everyday life of all road users, regardless of their preferred mode of transportation: whether they sit behind the wheel of their own car, walk, or let the public service carry them about their business, traffic lights will be regulating their movements and those of others around them (most noticeably, of those in front).

It is therefore natural that traffic signals should garner so much attention: they are perceived (only \emph{sometimes} unfairly) as a major source of delay and frustration to drivers, and the tantalising idea of an intelligent traffic control system often comes to identify, in the general public fantasy, with the very notion of an \emph{Intelligent Transport System}.

In fact, a history of case studies shows that wherever public money has been invested into the development and maintenance of a signalisation system tailored to the transportation needs of a community, the returns have invariably surpassed expenditures by far \citep{koonce2008traffic}.
\todo{comment on WHAT parameters were considered to determine "returns"}

Carefully planned signalisation allows a more efficient use of the existing road infrastructure,
minimising the stress suffered by drivers as well as the risk of accidents,
favouring public transport and improving air quality, with a positive impact on virtually every aspect of life in a modern city.

\section*{About Notation}
A quick glossary of the relevant variables is provided below, alongside the units of each dimensional quantity.

For a leaner presentation of the model, subscripts referring to
topological elements may be dropped to simplify notation.

\subsubsection*{Network Topology}
\begin{tabu} to \textwidth {X[3,c] X[1,c] X[6,l]}
$i,j \in \mathrm{N}$ & & nodes (junctions) \\[2pt]

$A,B \in A \subseteq \mathrm{N} \times \mathrm{N} $ & & arcs (lane groups) \\[2pt]

$\left( \mathrm{N}_{a}^{-} \; , \;\mathrm{N}_{a}^{+} \right) = a$ & & tail and head nodes of arc $a$ \\[2pt]

$\mathrm{A}_{i}^{+} = \left\lbrace a \in \mathrm{A} \, | \, \mathrm{N}_{a}^{-} = i \right\rbrace $ & & forward star of node $i$ (outgoing arcs)\\[2pt]

$\mathrm{A}_{i}^{-} = \left\lbrace a \in \mathrm{A} \, | \, \mathrm{N}_{a}^{+} = i \right\rbrace $ & & forward star of node $i$ (incoming arcs)\\[2pt]

$y,z \in \mathrm{Y}$ & & manoeuvres\\

\end{tabu} 


\subsubsection*{Signal Phases}
\begin{tabu} to \textwidth {X[3,c] X[1,c] X[6,l]}
$p,q \in \mathrm{P}_{j}$ & & signal phases at junction $j$\\[2pt]

$\mathrm{A}_{p} \subseteq \mathrm{A}_{j}^{-} $ & & lane groups open during phase $p$ \\[2pt]

\end{tabu} 


\subsubsection*{Signal Timing}
\begin{tabu} to \textwidth {X[3,c] X[1,c] X[6,l]}
$ t_j^C $ & s & cycle time at junction $j$\\[2pt]
$ t_p   $ & s & nominal duration of phase $p$\\[2pt]
$ g_a   $ & s & effective green duration for arc $a$\\[3pt]
$ \displaystyle \gamma_a = \frac{g_a}{t_j^C} $ & & effective green share of arc $a$\\[2pt]
$ t_j^L $ & s & time lost per cycle at junction $j$\\[2pt]
$ t_j^O $ & s & offset of junction $j$ \\[2pt]
\end{tabu} 



\subsubsection*{Demand and Supply}
\begin{tabu} to \textwidth {X[3,c] X[1,c] X[6,l]}
$ q_{a} $ & veh/s & demand flow on arc $a$ \\[2pt]
$ \hat{q}_a $ & veh/s & saturation flow of arc $a$ \\[2pt]
$ \displaystyle \phi_a = \frac{q_a}{\hat{q}_a} $ & & flow ratio on arc $a$ \\[2pt]
$ \displaystyle \chi_a = \frac{\phi_a}{\gamma_a} $ & & saturation on arc $a$ \\[2pt]
\end{tabu} 

\subsubsection*{Performance Indicators}
\begin{tabu} to \textwidth {X[3,c] X[1,c] X[6,l]}
$ t_a^Q $ & s & queue clearance time on arc $a$ (per cycle)\\[2pt]
$ \omega_a^{stop}$ & & share of $q_a$ stopping at or before $\mathrm{N}_{a}^{+}$ \\[2pt]
$ \omega_a^d $ & s & average delay of arc $a$ \\[2pt]
$ $ & & \\[2pt]
\end{tabu}