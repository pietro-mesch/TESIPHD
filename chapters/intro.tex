\chapter*{Introduction}
The fundamental role of \emph{traffic signals} is to equitably and efficiently administer the right of way amongst conflicting streams of road users.

Since the first sporadic appearances around the turn of the 20th century, traffic lights have become a ubiquitous feature in the everyday life of all road users, regardless of their preferred mode of transportation: whether they sit behind the wheel of their own car, walk, or let the public service carry them about their business, traffic lights will be regulating their movements and those of others around them (most noticeably, of those in front).

It is therefore natural that traffic signals should garner so much attention: they are perceived (only \emph{sometimes} unfairly) as a major source of delay and frustration to drivers, and the tantalising idea of an intelligent traffic control system often comes to identify, in the general public’s fantasy, with the very notion of an \emph{Intelligent Transport System}.

In fact, a history of case studies shows that wherever public money has been invested into the development and maintenance of a signalisation system tailored to the transportation needs of a community, the returns have invariably surpassed expenditures by far [???16].
%??? comment on WHAT parameters were considered to determine "returns".