\chapter{Obiettivi Intelligenti}
Definizione degli obiettivi dell'ottimizzazione classica e di quella sostenibile.

Formalizzazione delle funzioni di costo:
l'indice di coda (integrale tempo in coda per veicoli)/tempo passato nel corridoio fa cacare
$$
\frac{\sum_{a \in C} \sum_{t \in T}{queu_{at}}}{\sum_{a \in C} \sum_{t \in T}{n_{at}}}
$$

non è reattivo soprattutto se gli archi sono lunghi perché il tempo in coda è comunque poco rispetto al totale

ci vorrebbe tempo per km percorso, ma come faccio a calcolare lo spazio percorso?

Numero di fermate non è male, calcolato come integrale dell'aumento della QUEU in ogni intervallo, corretto aggiungendo l'eventuale deflusso se la coda alla fine dell'intervallo non si è azzerata.
Questo comporta degli errori (da quantificare esattamente?) negli intervalli in cui la coda comincia o finisce.
$$
\sum_{a \in C} \sum_{t \in T} QUEU_{a,t} - QUEU_{a,t-1} + OFLW_{a,t} \Delta t
$$


Il tutto è da normalizzare rispetto al numero di veicoli entrati su qualsiasi arco (forse meglio fare entrambe le somme arco per arco, più chiaro).

Altro obiettivo utile sarebbe guardare la congestione degli archi.
Prendendo la congestione media si auto-pesano di più gli archi corti, il che avrebbe senso.
