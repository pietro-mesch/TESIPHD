\chapter{Smart Objectives}
Definizione degli obiettivi dell'ottimizzazione classica e di quella sostenibile.

\subsection{Fundamental Quantities}
The following quantities are calculated on the corridor over the entire simulation, and represent the reference quantities for calculation of key performance indicators.

\subsubsection*{Section and Corridor Total}
The \emph{section total} is defined as the integral of the inflow to a given section of the corridor $a \in \corridor$, obtained piecewise in this case, as the total number of vehicles entered during each interval of the simulation window:
\eq[.]{eq:stotal}{
\total{a} \; = \sum_{\tau \in \simspan} \inflow_{a,\tau} \simintd
}

The \emph{corridor total} gives an aggregate measure of how frequented the corridor is on the whole: it does not carry information on which sections are busier, but accounts for all vehicles that accessed \emph{any} section during the simulation.

It is obtained as the cumulative total over all corridor sections, according to
\eq[.]{eq:ctotal}{
\total{\corridor} \; = \sum_{a \in \corridor} \total{a}
}

Notice that $\total{a}$ implies no distinction based on whether the flows are coming from the previous section of the corridor or from a cordon arc, therefore vehicles travelling on more than one section are counted several times. 
This reflects the fact that the corridor is being used \emph{more} if vehicles travel a greater portion of it than if they only were to use one section.

The total inflow index $\total{\corridor}$ covers an important role as a \emph{checksum}, since it ensures that any improvements in other cumulative indices are not really due to the corridor accepting fewer vehicles because of a deterioration in the traffic conditions.


\subsection{Dynamic Weighting}
Formalizzazione delle funzioni di costo:
l'indice di coda (integrale tempo in coda per veicoli)/tempo passato nel corridoio fa cacare
$$
\frac{\sum_{a \in C} \sum_{t \in T}{queu_{at}}}{\sum_{a \in C} \sum_{t \in T}{n_{at}}}
$$

non è reattivo soprattutto se gli archi sono lunghi perché il tempo in coda è comunque poco rispetto al totale

ci vorrebbe tempo per km percorso, ma come faccio a calcolare lo spazio percorso?


Numero di fermate non è male, calcolato come integrale dell'aumento della QUEU in ogni intervallo, corretto aggiungendo l'eventuale deflusso se la coda alla fine dell'intervallo non si è azzerata.
Questo comporta degli errori (da quantificare esattamente?) negli intervalli in cui la coda comincia o finisce.
$$
\sum_{a \in C} \sum_{t \in T} QUEU_{a,t} - QUEU_{a,t-1} + OFLW_{a,t} \Delta t
$$

Il tutto è da normalizzare rispetto al numero di veicoli entrati su qualsiasi arco.


Altro obiettivo utile sarebbe guardare la congestione degli archi.
Prendendo la congestione media si auto-pesano di più gli archi corti, il che avrebbe senso.
Come indicatore della congestione, dal GLTM prendiamo la lunghezza della coda peggiore nell'intervallo rispetto all'arco $QUEL\in[0,1]$ quindi come funzione di costo
\eq{e:maxqueue}{
Q_{a}^{MAX} = \sup\{QUEL_{a,t}|t\in T\}
}

Prendendo invece la lunghezza totale della coda?