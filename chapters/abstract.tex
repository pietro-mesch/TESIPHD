\chapter*{Abstract}
This dissertation focusses on the implementation of a Real-Time Simulation-Based Signal Coordination module for arterial traffic, as proof of concept for the potential of integrating a new generation of advanced heuristic optimisation tools into Real-Time Traffic Management Systems.
The endeavour represents an attempt to address a number of shortcomings observed in most currently marketed on-line signal setting solutions.
It is \emph{unprecedented} in its use of a Genetic Algorithm coupled with Continuous Dynamic Traffic Assignment as solution evaluation method, only made possible by the recently presented parallelisation strategies for the underlying algorithms.

Within a fully functional traffic modelling and management framework, the optimiser is developed independently, leaving ample space for future adaptations and extensions, while relying on the best available technology to provide it fast and \emph{realistic} solution evaluation based on reliable real-time supply and demand data.
The optimiser can in fact operate on high quality network models that are well calibrated and always up-to-date with real-world road conditions; rely on robust, multi-source network wide traffic data, rather than being attached to single detectors; manage area coordination using an external simulation engine, rather than a na\"ive flow propagation model that overlooks crucial traffic dynamics; and even incorporate real-time traffic forecast to account for transient phenomena in the near \emph{future} to act as a feedback controller.

Results clearly confirm the efficacy of the proposed method, by which it is possible to obtain relevant and consistent corridor performance improvements with respect to widely known arterial bandwidth maximisation techniques under a range of different traffic conditions. \\
The computational efforts involved are already manageable for realistic real-world applications, and future extensions of the presented approach to more complex problems seem within reach thanks to the load distribution strategies already envisioned and prepared for in the context of this work.