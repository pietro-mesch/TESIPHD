
\chapter{Results} \label{c:results}


\section{Corridor Performance Optimisation} \label{r:kpi}
These tests are performed on randomly generated 8 junction corridors with side flows

The test scenario covers the optimisation phase: models represent the corridor sub-networks, as would be produced for optimisation. Traffic flow profiles entering the cordon arcs and hence the corridor, as well as turn rates are assumed pre-calculated in the DTA phase, as detailed in section \ref{s:rollingtre}.

\todo{Demonstrate capability to improve over the SlackBand solution.}

\subsection{Stable Subcritical Demand}

Under \emph{sub-critical} traffic conditions, i.e. where no corridor link is subjected to demand flows over 80\% capacity \todo{simbol}

\fig{htbp}{PIX/undoguy.png}{f:subcrit}{\todo{ comparison of kpi curves, show very good improvements} }{width=0.4\textwidth}

The solutions match the geometrical solution

\fig{htbp}{PIX/undoguy.png}{f:similar}{ \todo{The solutions match the geometrical solution} }{width=0.4\textwidth}

\subsection{Stable Supercritical Demand}
Under \emph{super-critical} traffic conditions, i.e. where several arcs operate above their saturation capacity and queues \emph{are bound to form} at at least one junction along the corridor.

cold start
\fig{htbp}{PIX/undoguy.png}{f:cold start}{ \todo{cold start graph} }{width=0.4\textwidth}

rolling

\fig{htbp}{PIX/undoguy.png}{f:rolling start}{ \todo{rolling start graph} }{width=0.4\textwidth}


\subsection{Anti-Spillback Gating}
\todo{Network with a short arc that will spill back if people are allowed on it too fast. Junction just before it must not be blocked as a lot of people exit there}.

\subsubsection*{Eight Junction corridor with side flows} \label{s:sync8}
\todo{Describe the features of sync8 and why it is representative of the problem.}

\todo{CUT?}


\section{Computational Performance}



\subsection{Look-Ahead Window Size} \label{res:windowsizing}

\todo{Considerations upon computation time issues and applicability, scaling.}
Different corridor lengths.

\todo{can be used for planning!}



\section{Bi-directional Slack Bandwidth}
slackband section \ref{s:slackband}.

\todo{Demonstrate how the two way corridor actually performs much better with the slack band solution:
\begin{itemize}
\item plan network
\item recycle band optimiser (from home)
\item draw picture \ref{f:slackvhard}
\item make graph
\end{itemize}
}

\fig{htbp}{PIX/undoguy.png}{f:slackvhardres}{ Bar graph with 70-30\% main:return optimised slackband vs maxband , SsShMain, QsQhMain ... TsThReturn}{width=0.4\textwidth}

Slack Bandwidth section \ref{s:slackband} show it is much better than reg band for similar "progression" objectives (travel times) for a two-way band max problem with weights proportional to flows. \todo{rant about how novel this is.}

\section{Algorithm Parameters}

This section presents results of the preliminary studies aimed at determining an optimal choice of parameters for the Genetic Algorithm.
\todo{Genetic Algorithm parameters, population priming, dynamic weighting effects}

\subsection{Population Size}
\fig{htbp}{PIX/undoguy.png}{f:popsize}{ Effect of population size on one problem }{width=0.4\textwidth}


Might serve as a general rule of thumb for practical application.
\fig{htbp}{PIX/undoguy.png}{f:popreq}{ Optimum population size vs number of junctions }{width=0.4\textwidth}

\subsection{Population Priming} \label{s:poppriming}

\fig{htbp}{PIX/undoguy.png}{f:priming}{ Show effect of initial population mix on convergence. }{width=0.4\textwidth}
