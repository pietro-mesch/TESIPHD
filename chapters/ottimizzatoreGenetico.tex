\chapter{A Real-Time Forecast-Based Genetic Algorithm}

The foremost aim of this work is to exploit the versatility and speed of advanced macroscopic traffic simulation to bring forth a heuristic approach capable of improving signal plans \emph{in real time}.
In this, it represents an attempt to bring together the best of most signal setting approaches described so far:
\begin{description}
\item[adaptive] aiming at real-time operation, it is hoped it might guarantee a degree of adaptivity so far only expected of \todo{adaptive online signal setting approaches described in ref?? using much simpler models}
\item[accurate] \todo{comprehensive model including real time data and less dependant on single sensors (see balmanual in case it says anything interesting about the stability of optima}
\item[heuristic] using heuristics avoid the simplifications required to formulate analytical approaches without compromising on accuracy in evaluating the outcomes
\item[gating] by evaluating forecast traffic conditions over a look-ahead window, it should behave more like a feedback controller, accounting at least for the short-term consequences of its decisions.
\todo{it should stay away from greedy solutions that may maximise the immediate efficiency of junctions while disregarding even the short-term effects which might include an increase in congestion}
\end{description}

This chapter presents the approach in detail, beginning with an introduction to the TRE macroscopic simulation engine, touching upon the most relevant features of the underlying model and framework, and finally describing the genetic algorithm implementation.



%%%%%%%%%%%%%%%%%%%%%%%%%%%%%%%%%%%%%%%%%%%%%%%%%%%%%%%%%%%%%%%%%%%%%%%%%%%%%%%%%%%%%%%%%%%%%%%%%%%%%

\section{TRE simulation engine and Optima framework}
The proposed optimisation method relies on the macrosimulation engine known as TRE, based on the eponymous Dynamic User Equilibrium assignment algorithm.
TRE lies at the core of the \todo{optima, reliability, installations around the world}.\\
This section aims to illustrate its fundamental principles of operation, in order to clarify how they might affect the optimisation and better understand the role of the simulation engine within the architecture.

\todo{introduce the idea of DTA as simulation, show route choice/network loading iterations}

\fig{htbp}{PIX/undoguy.png}{f:dta}{\todo{General Dynamic Traffic Assignment} \citep{yperman2007link} p22}{width=0.4\textwidth}

\todo{Traffic is considered to be a continuum, both as far as vehicle movements and trip-maker decisions are concerned}

Trip planning and route choice models are extremely relevant to the accuracy of a DTA and to the computational effort involved.
However, the representation of traffic propagating and interacting with the network and signals, and all related phenomena within the scope of the present study, falls upon the Dynamic Network Loading model.\\
The most suitable macroscopic model for the task at hand is the General Link Transmission Model, which will be analysed in further detail over the next few sections.

\subsection{General Link Transmission Model}
The \emph{General Link Transmission Model}, henceforth referred to as GLTM, is a model for continuous dynamic network loading: it can be used to determine time-dependent link flows $\flow_{a,\tau}$ and travel times $\traveltime{a,\tau}$ given the time-dependent route flows.

It is built upon the representation of traffic as a partially compressible one-dimensional fluid flowing through the network according to the principles of \emph{Kinematic Wave Theory} (KWT), as developed independently by \cite{lighthill1955kinematic} and \cite{richards1956shock}.

Its origins can be traced back to the \emph{Cell Transmission Model} first presented in relation to highway traffic by \cite{daganzo1994cell} and shortly after applied to network traffic in \citep{daganzo1995cell}: it was the first dynamic traffic representation based on hydrodynamic theory, and borrowed heavily from that field, as is most evident from the cell-based space discretisation of the road network adopted directly from computational fluid dynamics.

The need for cell discretisation was eliminated in the \emph{Link Transmission Model} presented by \cite{yperman2005link}. This innovative approach allowed dynamic network loading of large scale networks using a computationally efficient algorithm that only required calculations at intersection nodes, while solving for traffic propagation along whole links using kinematic wave theory: this allowed to do away with a significant complexity factor while still accurately modelling local flow restrictions and junction delays.

The original LTM, presented in full detail in \cite{yperman2007link}, rather simplified the wave propagation problem relying on the \emph{simplified} kinematic wave theory proposed by\citep{newell1993simplified}, whereby only two possible wave propagation speeds re contemplated: a forwards one for free-flowing traffic, and one for the congested flow states to propagate backwards.
This is a considerable approximation, as the relation between vehicle density, speed and the resulting flow is rather more complex in reality:
the instrument provided by kinematic wave theory to express such relations in general is the Density-Flow Fundamental Diagram, illustrated in section \ref{s:fundiag}.

While in truth the work of Yperman already improved on the simplified KWT approach to include any piecewise linear fundamental diagram, the GLTM presented in \cite{gentile2010general} was developed to extend the LTM formulation to any concave fundamental diagram, considerably improving the accuracy in representing delays due to congestion. The GLTM also uses time-varying capacity adjustments at nodes to accurately model conflicts at intersections and the so called \emph{spillback} of traffic states from downstream links to the relevant upstream ones.

The features of the supply model described in the next sections, together with the computational efficiency and the possibility to perform a DTA in high temporal resolution, make the General Link Transmission Model an optimal candidate for the present application.

\todo{what does ctm do for delay at junctions?}



\subsection{Link Model}
In the GLTM, traffic propagates along links according to kinematic wave theory.
As stated in section \ref{s:urbannetwork}, links are assimilated to weighted arcs of a directed graph, and as such are one-dimensional, one-directional and homogeneous along their length, stretching between locations $x^0$ (the tail node) and $x^1 = x^0 + \length$ (the head node): the actual link shape is inconsequential. As far as the arc model is concerned, there is no need to disambiguate arcs since they exist and are processed independently: arc subscripts can be dropped for ease of reading but are to be implied on all relevant quantities henceforth.

The traffic state at a specific location $x \in \left[x^0, x^1\right]$ is characterised by three macroscopic variables:
\begin{description}
\item[flow] $\flow_x$ : vehicles through the link section per unit time;
\item[density] $\density_x$ : average number of vehicles per unit length;
\item[speed] $\speed_x$ : average distance covered per unit time.
\end{description}
As is evident from their dimensions, only two of these quantities can be independent, and if two are known the third may be readily calculated using the relationship
\eq[.]{e:trafficstates}{\speed = \frac{\flow}{\density}}

\todo{link with next}

\subsubsection{Fundamental Diagram} \label{s:fundiag}
Kinematic wave theory assumes a functional relation between traffic density and flow, also known as the \emph{Fundamental Diagram} of traffic flow. It approximates the changes in the average behaviour of drivers as the road gets more crowded, and may take several forms, but it is invariably derived from the properties of the road, e.g. width, slope or parking. As such it is itself, conceptually, a property of the link, although it could also be made to depend on environmental factors and driver behaviour or be specific to a particular class of vehicles.

A generic fundamental diagram expresses the relationship between flow and density under \emph{stationary} traffic conditions, i.e. it is derived as an equilibrium condition between flow speed and available space taking the general form
\eq{e:fundiag_generic}{
q = f(k)
}
which may be represented on a Density-Flow graph like the ones shown in Figure \ref{f:fundiag}.
\fig{htbp}{PIX/undoguy.png}{f:fundiag}{Fundamental Diagram of a link, representing the functional relation between vehicular density and speed, resulting in different flow values for different congestion levels. Triangular form (left) and ??? form (right) are similarly shaped around the critical density value $\hat{\density}$, the jam density $\density_{jam}$ and the free flow speed $\vzero$.}{width=0.4\textwidth}

The shape of a fundamental diagram reflects different assumptions about ??? but there are a few key features that are shared by all formulations:
\begin{itemize}
\item when density approaches \emph{zero} the speed approaches the maximum value attainable on the link ($\vzero$) but the flow tends to zero;
\item maximum flow occurs at the \emph{critical density $\hat{\density}$} also referred to as the link capacity;
\item beyond capacity, further increase in vehicular density induces a speed penalty that causes the flow to decrease
\item when vehicles reach the \emph{jam density} $\density_{jam}$ they are packed as closely as possible, and come to a standstill;
\item for any flow state on the $k-q$ curve, the speed is given by the slope of the line connecting it to the origin;
\item the rising branch diagram (i.e. to the left of $\hat{\density}$) represents free flowing 
states, the descending branch represents congested states.
\end{itemize}

In a simple triangular diagram like the one shown in Figure \ref{f:fundiag} (left) the speed is assumed to remain constant at its maximum value for all hypocritical states, while above capacity it decreases linearly with density.
More detailed modelling of the involved phenomena may yield a diagram shape more similar to Figure \ref{f:fundiag} (right), where the speed is shown to decrease even in hypocritical conditions as the road gets more crowded, due e.g. to a certain variance in driving speed which, as more vehicles become involved, yields a higher chance of having a slow vehicle delaying all the others \emph{(hypocritical spacing)}.

In both cases it is assumed that as the density increases, the available space becomes insufficient to maintain safe distances between vehicles, causing drivers to slow down \emph{(hypercritical spacing)}.

If a model is to rely on the fundamental diagram to hold for non-stationary traffic as well, it must allow for vehicles to change speed instantly with infinite acceleration, as is the case with GLTM and in general with first-order implementations of KWT.


\todo{analysis of shapes given in ref tiddi}
\todo{higher order traffic phenomena such as the emergence of stop-and-go waves,
oscillatory congested traffic, or capacity drop (which is caused by the so-called hysteresis
effect, meaning that the sequence of traffic states follows a different path in the
fundamental diagram during breakdown to congestion than during recovery from
congestion (Tampère (2004)), are not considered}

\subsubsection{Lane Modelling}
\todo{compare with balance lane model.}

\subsubsection{Corridoio di ottimizzazione}
Definito come una serie di link.
I link attraversano un certo numero di intersezioni, alcune delle quali sono amministrate da un semaforo.
Fra l'inizio e il primo semaforo, fra i semafori, e dopo l'ultimo, i link sono accorpati in sezioni. I KPI vengono calcolati su di essi.

\subsubsection*{Return Corridor Definition}




\subsection{Node Model}



\subsection{Dynamic Traffic Assignment Algorithm}

DTA + DNL iterations


\fig{htbp}{PIX/DIAG-dta-scheme_550x320.png}{f:ass}{L'assegnazione di TRE}{width=0.8\textwidth}

Quantificare errori dovuti all'uso di TPRB fisse 

Data una finestra temporale di una certa lunghezza

\subsubsection{GLTM Input}

\subsubsection{Real Time Data Integration}

\subsubsection{GLTM Simulation Output} \label{s:output}
\todo{Specify what quantities are ultimately calculated in a TRE simulation and can be considered its output.}



%%%%%%%%%%%%%%%%%%%%%%%%%%%%%%%%%%%%%%%%%%%%%%%%%%%%%%%%%%%%%%%%%%%%%%%%%%%%%%%%%%%%%%%%%%%%%%%%%%%%%

\section{Genetic Algorithm}
\todo{Describe implementation, adaptations and tuning}
\subsection{Time Dependent Weighting of Cost Functions}


\subsection{Initial Population Seeding with Slack Bandwidth}
\todo{Describe how the slackband is implemented to initialise the population and obtain better solutions faster.}



%%%%%%%%%%%%%%%%%%%%%%%%%%%%%%%%%%%%%%%%%%%%%%%%%%%%%%%%%%%%%%%%%%%%%%%%%%%%%%%%%%%%%%%%%%%%%%%%%%%%%

\section{Online Heuristic Optimisation}
\todo{Describe the task at hand and the interaction between TRE/GA.}\\
Condizioni dei test di TRE:
\begin{enumerate}
\item TRE fa un equilibrio e produce TPRB per tutto il giorno: gli intervalli sono ben più lunghi di un ciclo di semaforo quindi le svolte sono mediate e dipendono solo da domanda e green share
\item TRE salva l'istantanea dei flussi per cominciare le simulazioni di ottimizzazione a rete carica
\item Gli offset vengono modificati e per ogni individuo si fa un caricamento al secondo su una finestra temporale ristretta, partendo dall'istantanea salvata
\end{enumerate}






Funzionamento generale, pregi e limitazioni rispetto all'applicazione come ottimizzatore.

Quale intervallo di risultati si può usare? Data l'incertezza e le approssimazioni, ha senso usare i risultati al secondo?

Non varrebbe la pena guadagnare tempo usando intervalli di 6s ?
\emph{Non importa, si guadagna tempo ma per ora il tempo non ci preoccupa. Semmai si fa una prova a posteriori una volta determinato quale combinazione degli altri parametri funziona meglio}.





