\chapter{Ottimizzazione con GLTM Genetico}
Perché è una buona idea?
E' una simulazione in grado di tenere conto di fenomeni complessi ma più veloce della micro.

L'infrastruttura contiene dati in tempo reale etc.

Condizioni dei test di TRE:
\begin{enumerate}
\item TRE fa un equilibrio e produce TPRB per tutto il giorno: gli intervalli sono ben più lunghi di un ciclo di semaforo quindi le svolte sono mediate e dipendono solo da domanda e green share
\item TRE salva l'istantanea dei flussi per cominciare le simulazioni di ottimizzazione a rete carica
\item Gli offset vengono modificati e per ogni individuo si fa un caricamento al secondo su una finestra temporale ristretta, partendo dall'istantanea salvata
\end{enumerate}

\fig{htbp}{PIX/DIAG-dta-scheme_550x320.png}{f:ass}{L'assegnazione di TRE}{width=0.8\textwidth}

Quantificare errori dovuti all'uso di TPRB fisse 

Data una finestra temporale di una certa lunghezza

\section{General Link Transmission Model}
Funzionamento generale, pregi e limitazioni rispetto all'applicazione come ottimizzatore.

Quale intervallo di risultati si può usare? Data l'incertezza e le approssimazioni, ha senso usare i risultati al secondo?

Non varrebbe la pena guadagnare tempo usando intervalli di 6s ?
\emph{Non importa, si guadagna tempo ma per ora il tempo non ci preoccupa. Semmai si fa una prova a posteriori una volta determinato quale combinazione degli altri parametri funziona meglio}.

\subsection{GLTM Simulation Output} \label{s:output}
\todo{Specify what quantities are ultimately calculated in a TRE simulation and can be considered its output.}

\section{Il modello in tempo reale}

\subsection{Corridoio di ottimizzazione}
Definito come una serie di link.
I link attraversano un certo numero di intersezioni, alcune delle quali sono amministrate da un semaforo.
Fra l'inizio e il primo semaforo, fra i semafori, e dopo l'ultimo, i link sono accorpati in sezioni. I KPI vengono calcolati su di essi.

\section{Tuning del modello}


\section{Tuning del Genetico}
Effetti dimensioni popolazione, aggiustamenti mutazione