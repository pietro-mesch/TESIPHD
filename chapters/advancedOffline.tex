\section{Advanced Offline Signal Planning}
The simple signal setting problems presented so far are quasi-convex, but more realistic traffic models that include and quantify global performance indicators such as total delay introduce an inherent non convexity, better addressed with the aid of heuristic methods.

With the increase in computing power availability, metaheuristics have seen a substantial rise in popularity as means to overcome the inherent limitations of analytical formulations: heuristic approaches to this class of problems involve the generation of a large — yet manageable, compared to the dimensions of the search space — number of candidate timing solutions, the effects of which are then simulated to evaluate their fitness. At each iteration, a variety of methods ranging from Genetic Algorithms to Simulated Annealing and Particle Swarm Optimisation can then be used to modify and combine the most successful solutions into a new set of candidates.

Such methods are particularly suited for solving obscure problems as they require no attempt to establish an explicit correlation between the control variables and the desired outcome. Rather, they rely on the assumption that if any relevant phenomena can be modelled with sufficient accuracy and a performance index can describe the degree of achievement of the optimization objectives, then the system can be led to evolve towards an optimal solution.

\fig{htbp}{pix/heur.png}{f:heuristic}{Conceptual information flow in a heuristic approach to signal optimisation}{width=0.65\textwidth}

It is therefore obvious that the model used to assess the fitness of candidate solutions should represent a sensible trade-off between speed and completeness: the real-world performance will inevitably be disappointing if the optimisation does not account for relevant traffic phenomena that were simplified out of the solution assessment, while on the other hand the need to evaluate huge numbers of candidate solutions calls for a lean and fast method to predict the outcome of a given timing plan. Furthermore, heuristics that depend heavily on the choice of initial conditions often use maximum bandwidth solutions as starting point in the search for minimum total delay, to shave off convergence time and increase the quality and applicability of solutions.
The present study takes full advantage of both features.

The heuristic optimisation approach has been taken most notably by the Transport Research Laboratory, the UK based institution that since \citep{robertson1969transyt} has been developing the TRAffic Network StudY Tool, which was born as a software tool to minimise stops along arterial roads while accounting for reasonably realistic vehicle behaviour, and was gradually extended to model ever more complex phenomena. 

Today, TRANSYT can handle pedestrian flows, optimise green shares as well as junction offsets and include actuated signals, all the while monitoring a custom set of network-wide performance indicators that can implement whatever policy the traffic administration desires. The optimisation relies on the availability of a complete transportation network model, possibly including detailed junction geometry, roughly corresponding to the requirements for the present application as described in Chapter \ref{c:optimiser}.
A range of search algorithms can be used to explore complex timing solutions, which are then evaluated using either micro– or macrosimulation models.
Earliest version of TRANSYT implemented a simple hill climbing algorithm that explored the non convex performance function by executing a predetermined set of short and long steps to vary each control variable in both directions alternatively. At each step, the changed value of the control variable is kept if it improved the performance index. 

\cite{park1999traffic} introduced a traffic signal optimization program for oversaturated intersections consisting of two modules: a genetic algorithm optimizer and mesoscopic simulator. \cite{colombaroni2009optimization} devised a solution procedure that first applies a genetic algorithm and then a hill climbing algorithm for local adjustments; solution fitness being evaluated by means of a traffic model that computes platoon progression along the links, simulating their combination and possible queuing at nodes through analytical delay formulations. The model was also extended to design optimal signal settings for a synchronised artery with predetermined rules for dynamic bus priority.

Metaheuristics often see applications in traffic signal engineering that reach beyond ordinary signal planning, and have more than once played an important role in research by aiding the formalisation of less intuitive correlations between signal settings and traffic behaviour. In \citep{gentile2009synchronization} a Genetic Algorithm was used to venture out into the yet uncharted territory of arterial synchronisation \emph{under heavy congestion and queue spillback}. To predict the outcome of candidate signal plans, the heuristic method relied on the General Link Transmission Model (see Chapter \ref{c:optimiser} and \cite{gentile2010general}), which implements the Kinematic Wave Theory to allow accurate simulation of traffic dynamics and model physical blockage of links, while requiring sufficiently short computation times to deal with the very large number of solutions to be evaluated.
In this case, the optimisation revealed a crucial difference between subcritical and supercritical flow conditions: while in the former case the optimal green wave is led as usual by the flow velocity, the same approach proves completely ineffective under supercritical conditions, which oppositely demand that the backwards propagating jam wave speed should set the pace of upstream signals, to ensure that the residual capacity of saturated links is fully exploited.

It must be noted that the level of detail taken into account when using metaheuristics comes at a \emph{heavy} cost in terms of computation time, which has \emph{so far} limited the functionality of this type of software to that of advanced - yet offline - planning tools; commonly accessible computing power being insufficient for true real time operation, advanced optimization suites are staying on top of the game by attempting to streamline the interactions between the development environment and the street-level equipment, e.g. providing offline optimisation based on real time readings and quick and simple deployment of new plans.

The present work aims to break new ground by exploiting the considerably shorter computation times brought to macroscopic traffic modelling by parallel computing, as presented in \citep{attanasi2015real}, and coupling them with a consolidated real-time traffic management environment to make a first step towards simulation-based heuristic optimisation based on real-time data. 

\todo{Conclude better?}
















