The idea of \emph{slack} bandwidth is an answer to the very strict definition of bandwidth given at the beginning of section \ref{s:bandmax}, according to which only the band running through all junctions counts for something, implying that:
\begin{itemize}
\item if one passing phase is particularly short, coordination between longer green phases may be disregarded: because of \req{e:hardband} the bandwidth is throttled to be at most as wide as the shortest phase;
\item in bi-directional optimisation, maximisation of the return band may prioritise a very narrow band that \emph{just} makes it through all junctions (possibly degrading the main band significantly) over a very wide band divided in two or more chunks.
\end{itemize}

To avoid such inconveniences, which are intrinsic in the definition of what will henceforth be referred to as the \emph{canonical} bandwidth $\hardband{\corridor}$, the slack bandwidth paradigm attempts to describe the overall \emph{"progressivity"} of the corridor along its whole length, by considering the sum of the individual green bands leading up to and following \emph{any} of the corridor junctions.

Rather than a length of time (the width of $\hardband{\corridor}$ on a T-D graph) the slack bandwidth has dimensions \units{L \cdot T} (an area on a T-D graph) i.e. it is the product of the time during which a vehicle may enter each section of the corridor and the distance it will travel unhindered as a result. If the times are normalised with respect to the cycle time, the slack band becomes a \emph{probability $\times$ distance} product (just as the canonical bandwidth would represent the overall chance of travelling the whole corridor without stopping).

The formalisation of this idea is much simpler than it may sound at first. 
Consider a junction $j$ somewhere along corridor $\corridor$: the \emph{forwards slack progression band} $\fslackband{j}$ is the integral of: the distance $l_j$ that may be travelled without stopping, with respect to the time $t$ at which a vehicle leaves from $j$; the former obviously a function of the latter which may be expressed as $l_j(t)$.

Using a compact definition of the \emph{through} phase at $j$ as the interval during which the corresponding manoeuvres are open
\eq[,]{e:gint}{\gint{j} = \left[ \gini{j} \; , \;\: \gini{j} + \gren{j} \right]}
where $\gini{j}$ is the beginning time of the through phase at $j$ and $\gren{j}$ its duration, as before, the forward band can be expressed as
\eq[.]{e:slackint}{
\fslackband{j} = \int_{\gint{j}} l(t) \; \mathrm{d}t}

The integral in \req{e:slackint} clearly formalises the definition of slack bandwidth illustrated in Figure \ref{f:slackdef}.a but it's not practical to compute, and does not provide an explicit form for $l(t)$.

Consider therefore the interval $\gint{ji}$ during which a vehicle that \emph{left} $j$ during $\gint{j}$ may drive through a \emph{subsequent} junction $i \geqslant j$. As the vehicles progress along the corridor, only the ones that reach each junction $i$ during the corresponding through phase $\gint{i}$ can proceed without stopping.

Now, the interval over which vehicles that left $i$ during $\gint{i}$ reach $i+1$ can be expressed as 
\eq[.]{e:ginth}{
\ginth{i} = \left[\inf \gint{i} + t_i \, , \, \sup \gint{i} + t_i \right]}
with a simple forward translation to account for the travel time $t_i$ of the relevant corridor section, whence the passing band can be shown to gradually narrow down by intersection with each subsequent green phase
\eq[.]{e:slacknarrowing}{
\gint{j \, i} = \gint{i} \cap \ginth{j \, i-1}}

Finally, using $|\gint{}|$ to indicate the length of a passing interval, the forward slack band can be calculated recursively from any junction $j$ to the end of the corridor:
\eq[,]{e:slacksum}{
\fslackband{j} = \sum_{i \corridor \geqslant j}
\left| \gint{ji} \right| \cdot \length_{i}
}
considering that at the reference junction the passing interval \emph{is} the green phase $\gint{j\, i=j} = \gint{j}$.

\fig{htb}{PIX/slackband.png}{f:slackdef}{ 
The \emph{forward slack band} definition on T-D diagrams expressed in \textbf{(a)} as the integral \req{e:slackint} and in \textbf{(b)} as the discrete sum \req{e:slacksum}.}{width=\textwidth}

Plainly following the specular process back to the beginning of the corridor, it is possible to calculate the \emph{backwards slack progression band} $\bslackband{j}$, to quantify the chances of a vehicle reaching junction $j$ unhindered by red lights. It is also plain that the subsequent applications of \req{e:slacknarrowing} may well yield an empty intersection before the end or the beginning of the corridor are reached: this is not an issue, as the value of this \emph{something-is-better-than-nothing} approach lies exactly in the ability to consider \emph{any} length that can be travelled without stopping. Some computation time can be saved by checking for this condition and stopping the recursive process \req{e:slacksum} as soon as all vehicles that left $j$ have stopped at $i$ (or, going the other way, as soon as none of the vehicles leaving $i$ will reach $j$ without stopping).

The \emph{total} slack band for a given corridor is the normalised sum of the forwards and backwards bands calculated at each junction:
\eq[.]{e:slackband}{
\slackband{\corridor} = \frac{1}{|\junset_\corridor|} \sum_{j \in \corridor} \bslackband{j} + \fslackband{j}
}
With the formalisation complete, it is worth noting the following aspects about the new metric, also illustrated in Figure \ref{f:slackvhard}:
\begin{itemize}
\item normalisation implies that the method favours letting vehicles onto \emph{longer} arcs, which maximise the product in \req{e:slacksum}, rather than short ones, which are more vulnerable to spillback;
\item if a perfect, continuous green wave can be obtained along the whole corridor, the result is identical to the canonical bandwidth value multiplied by the length of the corridor 
$\slackband{\corridor} = \hardband{\corridor} \cdot \length_\corridor$ (this requires that all green phases also have the same length);
\item in all other cases, the slack band value is \emph{strictly greater} than 
$\hardband{\corridor} \cdot \length_\corridor$ as it factors in all fringes and partial bandwidths that \req{e:milp7} necessarily excludes, which is the main point of this metric.
\end{itemize}

\fig{htbp}{PIX/slackvshard.png}{f:slackvhard}{A comparison of the results obtained by optimising the canonical progression bandwidth (left) and the slack band metric (right) on T-D diagrams.\\
Bands crossing from the top-left to the bottom-right are travelling in the main direction, the others are in the secondary direction, going through the junctions in reverse order.
Darker bands on the slack band diagram are in common between a higher number of junctions, the darkest corresponding to the canonical bandwidth and all others to the fringes and partial green waves that the new metric allows to weigh in.}{width=0.7\textwidth}

With given offsets and signal programs, the computation of this metric is almost instantaneous for any conceivable real-world traffic arterial, allowing to find an optimal solution in seconds using a stochastic search method as described in section \ref{s:seeding}. Its effectiveness confirmed by the results presented in \todo{ReF}, this method was used to find ideal initial conditions for the real-time traffic coordination module presented in this work.