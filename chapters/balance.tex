\chapter{The Benchmark}
\todo{This chapter introduces the Genetic Optimiser currently shipped with many OPTIMA systems, which will used as a sparring partner to evaluate the performance of the TRE based optimiser.}

\begin{itemize}
\item Il modello è meglio o peggio di TRE ?
\item Perché usare TRE che è molto più lento?
\item come funziona BAL e cosa può fare?
\end{itemize}

Balance ha un modellino \emph{mesoscopico} tipo gltm (n realtà è proprio come GLTM al secondo) e fa un taglio delle reti intorno alle junction che vuole controllare. Sugli ingressi alle sottoreti (una per ogni junction) usa profili di flusso costanti, ma se può usa i flussi uscenti di una junction per determinare i flussi entranti in una a valle (propagazione).

Il modello viene usato per ricavare le funzioni di costo FERMATE, LUNGHEZZA CODE (in realtà numero di veh in coda) e PERDITEMPO.
Lui vede le code come F-E perché i suoi archi sono in realtà le corsie di svolta, ed usa una lunghezza MASSIMA per le code.

Punti di forza di Balance:
\begin{itemize}
\item Fa tante intersezioni
\item E' veloce
\item Aggiusta anche le durate degli stage
\end{itemize}

Punti deboli:
\begin{itemize}
\item Non è detto che le intersezioni si parlino tanto bene
\item Non guarda avanti
\item non vede l'arco ma solo l'approccio: forse una volta che la coda ha raggiunto il sensore per lui tutte le situazioni sono uguali, e sotto carico non gli cambia più niente
\item Probabilmente tende a massimizzare la capacità dove è più richiesta, favorendo lo scorrimento ma provocando un comportamento "ingordo" che crea problemi a valle.
\end{itemize}

\section{The Performance Index}
For each sub-network $\arcset_j$ around a controlled junction $j$, Balance evaluates a composite performance index based on vehicle delay $D$, number of stops $S$ and queue lengths $Q$ which takes the following form:
\newcommand{\qjxj}{\mathsmaller{\mathsmaller{(\flows{j},\controlvars{j})}}}
\eq[,]{e:balancepi}{
\balpi{j}\qjxj = \sum_{a \in \bstar{j}}
\weight{a}{D} D_a\qjxj \: + \:
\weight{a}{S} S_a\qjxj \: + \:
\weight{a}{Q} Q_a\qjxj
}
where the $\weight{a}{}$ terms are arc specific weights on the value of each component performance function, while $\flows{j}$ represents a generic vector of flow values at the junction and $\controlvars{j}$ the set of decision variables.

\subsubsection{Balance Requirements} \label{s:balreq}
Serve un detector con channel unico per ogni corsia interessante di approccio all'intersezione.

\subsection{Balance Settings} \label{s:balset}


