\chapter{The Benchmark}
\todo{This chapter introduces the Genetic Optimiser currently shipped with many OPTIMA systems, which will used as a sparring partner to evaluate the performance of the TRE based optimiser.}

\begin{itemize}
\item Il modello è meglio o peggio di TRE ?
\item Perché usare TRE che è molto più lento?
\item come funziona BAL e cosa può fare?
\end{itemize}

Balance ha un modellino \emph{mesoscopico} tipo gltm (n realtà è proprio come GLTM al secondo) e fa un taglio delle reti intorno alle junction che vuole controllare. Sugli ingressi alle sottoreti (una per ogni junction) usa profili di flusso costanti, ma se può usa i flussi uscenti di una junction per determinare i flussi entranti in una a valle (propagazione).

Il modello viene usato per ricavare le funzioni di costo FERMATE, LUNGHEZZA CODE (in realtà numero di veh in coda) e PERDITEMPO.
Lui vede le code come F-E perché i suoi archi sono in realtà le corsie di svolta, ed usa una lunghezza MASSIMA per le code.

Punti di forza di Balance:
\begin{itemize}
\item Fa tante intersezioni
\item E' veloce
\item Aggiusta anche le durate degli stage
\end{itemize}

Punti deboli:
\begin{itemize}
\item Non è detto che le intersezioni si parlino tanto bene
\item Non guarda avanti
\item non vede l'arco ma solo l'approccio: forse una volta che la coda ha raggiunto il sensore per lui tutte le situazioni sono uguali, e sotto carico non gli cambia più niente
\item Probabilmente tende a massimizzare la capacità dove è più richiesta, favorendo lo scorrimento ma provocando un comportamento "ingordo" che crea problemi a valle.
\end{itemize}

\section{Traffic Model} \label{s:balancetrafficmodel}
Balance builds a two-level traffic model:
\begin{description}
\item[Macroscopic level]: \todo{based on OD and assignment, gives the inflow values to the junction subnetworks, runs once per optimisation window}
\item[Mesoscopic level]: \todo{second-by-second flow model, generates flow profiles based on macro flows and decision variables, computes performance indices}
\end{description}

The macroscopic model performs an incremental traffic assignment, consisting of a series of partial assignments of increasing shares of the expected demand.\\
Each iteration assigns the corresponding share of flows according to a path search that accounts for any delay due to the flows already assigned, finally obtaining an estimate of the total flows on all arcs of the network. These may be further refined using iterative corrections if real flow measures are available.

\todo{OPTIMA may be plugged in here instead.}

The mesoscopic model begins by computing an inflow profile $\iflob{a,\tau}$ for each approach to a junction, determining in detail the expected dynamics of the relevant inflows over the course of a signal cycle. \todo{ref scoot and formalise?} 

The notion of \emph{arc} in this context corresponds to that of \emph{lane group}, with the tail corresponding to the detector position (in accordance to the operational requirements specified in section \ref{s:balreq}), and the head with the signalised stop line.
The flow entering the arc during interval $\tau$ is assumed to reach the stop line in a future interval $\tau'$ by propagating forwards at \emph{constant speed}.
The exit flow $\eflob{a}$ accounts for inflow, queue, lane capacity and signal timing, but not for any downstream effects:
\eq[.]{e:balpropagation}{
\text{with} \; \tau' \geq \tau \; | \; \tau_0 + \length_a / \vzero_a \in \tau'
\qquad
\eflob{a,\tau'} = \left\lbrace
\begin{array}{ll}
0 \qquad 	&	\text{if lane group is closed}	\\
\max{\left(
\satflow_a \: , \:
\tfrac{\nveh{a,\tau}}{\simintd} + \iflob{a,\tau}'
\right)}	&	\text{if lane group is open}		\\
\end{array}
\right.
}

The term $\iflob{a,\tau}'$ is used to model \todo{platoon dispersion along the link}
\eq{e:dispersion}{
\iflob{a,\tau} = \frac{1}{2t+1} 
\sum_{\tau''=\tau-t}^{\tau+t} \iflob{a,\tau''} 
\quad
\text{where} \quad t \in \mathds{N} = \length_a \backslash \vzero_a \simintd
}
\todo{this dispersion would average anything out if tt = cycle/2}

\todo{describe propagation and define exit flow $e$}



\section{Performance Index}
For each sub-network $\arcset_j$ around a controlled junction $j$, Balance evaluates a composite performance index based on vehicle delay $D$, number of stops $S$ and queue lengths $Q$ which takes the following form:
\newcommand{\qjxj}{\mathsmaller{\mathsmaller{(\flows{j},\controlvars{j})}}}
\eq[,]{e:balancepi}{
\balpi{j}\qjxj = \sum_{a \in \bstar{j}}
\weight{a}{D} D_a\qjxj \: + \:
\weight{a}{S} S_a\qjxj \: + \:
\weight{a}{Q} Q_a\qjxj
}
where the $\weight{a}{}$ terms are arc specific weights on the value of each component performance function, while $\flows{j}$ represents a generic vector of flow values at the junction and $\controlvars{j}$ the set of decision variables. 
\todo{plus there's an undocumented arc global weight }

The delay, stops and queue terms are calculated by the mesoscopic model as detailed in the following sections.

\subsubsection*{Balance Delay $D_a$}
Considering that the mesoscopic model regards lane groups as arcs, the total delay for a signal group over a given time window is approximated by the integral of time spent by vehicles on the corresponding arcs. 

This is readily obtained from the arc occupancy profile during the desired time window, in turn obtained as the difference between the entry and exit cumulative flows (analogous to those defined for the GLTM in \todo{ref}) which in this case can be taken to be the integrals of the entry and exit flow profiles described in section \ref{s:balancetrafficmodel}:
\eq[.]{eq:balcum}{
\fcum{a,t} = \sum_{\tau < t} \iflob{a,\tau} \simintd
\qquad \text{and} \qquad
\ecum{a,t} = \sum_{\tau < t} \eflob{a,\tau} \simintd
}
\todo{maybe define and use $\tau$ endpoints}

The difference is then taken to represent \emph{delayed} vehicles even though it will factor in \emph{all} vehicles travelling down the approach lane, even if they are not in fact hindered in any way, but given the nature of $\iflob{\tau}$ described by equation \eqref{e:balpropagation} and the short length accounted for (between the sensor and the stop line) this seems like an acceptable approximation. Finally, the \emph{total delay} is calculated over an interval $\simspan$ as:
\eq[.]{e:baldelay}{
D_{a,\simspan} = 
\sum_{\tau \in \simspan} \total{a,\tau} \simintd =
\sum_{\tau \in \simspan} \left( \fcum{a,\tau} - \ecum{a,\tau} \right) \simintd
} 

\subsubsection*{Balance Stops $S$}
The number of stops

\subsubsection*{Balance Queue Length $Q$}
Queues are

\subsubsection{Balance Requirements} \label{s:balreq}
\todo{Serve un detector con channel unico per ogni corsia interessante di approccio all'intersezione.}

If the synchronisation between signals is to be properly optimised, flow profiles must be collected on the forward star arcs of the controlled junctions in order to better estimate the arrival rates at downstream stop lines, \todo{which would otherwise be assumed constant as described in section \ref{???}}

\todo{To consider all relations between the signals and links in the network.
For this it is necessary to use the flow profiles of the links which lead out of the subnet, as
input variables for the signal groups which are located downstream. As described in chapter
2.1.4.3, the traffic flow profiles of the entry-links of a sub-net are initialized with equally
distributed profiles. If there is a traffic flow profile available for an exit-link of an adjacent}

\subsection{Balance Settings} \label{s:balset}


